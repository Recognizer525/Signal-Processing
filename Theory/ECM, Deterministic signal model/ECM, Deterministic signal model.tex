\documentclass[11pt]{article}
\usepackage[english,russian]{babel}
\usepackage[utf8]{inputenc}
\usepackage[a4paper, left=2.5cm, right=1.5cm, top=2.5cm, bottom=2.5cm]{geometry}
\usepackage{animate} 
\usepackage{graphicx}
\usepackage{amsmath}
\usepackage{longtable}
\usepackage{amssymb}
\usepackage{physics}
\usepackage{tikz}
\usepackage{comment}
\usepackage{animate} 
\usepackage{graphicx}
\usepackage{amsmath}
\usepackage{longtable}
\usepackage{amssymb}
\usepackage{physics}
\usepackage{tikz}
\usepackage{comment}
\usepackage{colortbl}
\usepackage{xcolor}
\usepackage[normalem]{ulem}
\usepackage{float}
\usepackage{wrapfig}
\usepackage{cancel}
\usepackage{mathtools}
\usepackage[most]{tcolorbox}
\usepackage[mathscr]{euscript}

\DeclarePairedDelimiter\ceil{\lceil}{\rceil}
 
\newcommand{\Expect}{\mathbb{E}}
\newcommand{\Var}{\mathcal{D}}
\newcommand{\Cov}{\mathsf{cov}}
\newcommand{\Norm}{\mathcal{N}}
\newcommand{\NormComplex}{\mathcal{CN}}
\newcommand{\Real}{\mathbb{R}}
\newcommand{\Int}{\mathbb{Z}}
\newcommand{\XSig}{\mathbf{x}}
\newcommand{\Ssig}{\mathbf{s}}
\newcommand{\Nsig}{\mathbf{n}}
\newcommand{\Rs}{\mathbf{R}_s}
\newcommand{\Rn}{\mathbf{R}_n}
\newcommand{\DK}{\mathbf{D}_{KL}}
\DeclarePairedDelimiterX{\infdivx}[2]{(}{)}{%
  #1\;\delimsize\|\;#2%
}
\DeclareMathOperator*{\argmax}{arg \, max}
\DeclareMathOperator*{\argmin}{arg \, min}
\DeclareMathOperator{\Det}{Det}
\newcommand{\infdiv}{D_{KL}\infdivx}
\newcommand\Fontvi{\fontsize{8.2}{7.2}\selectfont}
\newcommand\Fontvia{\fontsize{9}{8}\selectfont}
\newcommand\Fontvib{\fontsize{10.8}{9.6}\selectfont}
\newcommand\Fontvic{\fontsize{8.0}{7.0}\selectfont}
\newcommand{\myitem}{\item[\checkmark]}
%\newcommand{\myitem}{\item[\squares]}

\begin{document}
\begin{center}
\fontsize{20}{23}\selectfont \color{red}{\textbf{ECM, Детерминированная модель сигналов}}
\end{center}
\begin{center}
\fontsize{16}{20}\selectfont \color{teal}{\textbf{ \S1 Постановка проблемы}}
\end{center}
Предположим, у нас имеется линейная антенная решетка, состоящая из $L$ сенсоров. Решетка принимает волны, направленные из $M$ различных источников. Этим источникам соответствует вектор углов прибытия (DoA) $\theta$, практически не изменяющийся с течением времени. По итогам измерений, было получено $G$ снимков полученного сигнала, причем ввиду технических неполадок, связанных с сенсорами, большая часть таких снимков содержит помимо надежных данных ненадежные, которые в рамках данной задачи рассматриваются как пропуски. Пусть $X$ --- полный набор сигналов, полученных датчиками в моменты времени $t=\overline{1,G}$, $X_t$ соответствует сигналу в момент времени $t$, через $x$ и $x_t$ будем обозначать реализации полного набора сигналов и сигнала в отдельный момент времени $t$  соответственно. Ввиду наличия пропусков в данных, будем считать, что $X$ состоит из наблюдаемой части $X_o = \{X_{o_t}\}_{t=1}^G$ и ненаблюдаемой: $X_m = \{X_{m_t}\}_{t=1}^G$. Полученный сигнал является результатом следующей модели наблюдений:
\begin{equation}
X = A S + N,
\end{equation}
где $N=\{N_t\}_{t=1}^G$ соответствует набору шумов, связанных с датчиками в моменты времени $t=\overline{1,G}$, $S=\{S_t\}_{t=1}^G$ соответствует набору сигналов, испускаемых источниками в моменты времени $t=\overline{1,G}$, $A$ -- матрица управляющих векторов для равномерного линейного массива:
\begin{gather}
A(\theta) = \begin{bmatrix}
1&1&\dots&1\\
e^{-2j\pi \frac{d}{\lambda}\sin(\theta_1)}& e^{-2j\pi \frac{d}{\lambda}\sin(\theta_2)}&\dots&e^{-2j\pi \frac{d}{\lambda}\sin(\theta_M)}\\
\dots&\dots&\ddots&\vdots\\
e^{-2j\pi (L-1) \frac{d}{\lambda}\sin(\theta_1)}& e^{-2j\pi (L-1) \frac{d}{\lambda}\sin(\theta_2)}&\dots&e^{-2j\pi (L-1) \frac{d}{\lambda}\sin(\theta_M)}\\
\end{bmatrix}.
\nonumber
\end{gather}
Сигналы, испускаемые источниками, рассматриваются как детерминированные; в то же время шумы на сенсорах, предполагаются стохастическими: $N_t \sim CN(\mathbf{O}_{L \times 1}, \mathbf{\Lambda})$. Матрица $\mathbf{\Lambda}$ предполагается диагональной, т.е. шумы рассматриваются как некоррелированные. Для простоты дальнейших рассуждений введем также следующие величины:
\begin{itemize}
\item
$L_{o_t}$ --- число исправных сенсоров в момент времени $t$;
\item
 $L_{m_t}$ --- число неисправных сенсоров в момент времени $t$;
\item 
$A_{o_t}$ --- матрица, образованная теми строками матрицы $A$, которые соответствуют работающим сенсорам в момент времени $t$; 
\item
$A_{m_t}$ --- матрица, образованная теми строками матрицы $A$, которые соответствуют неисправным сенсорам в момент времени $t$;
\item
$\mathbf{\Lambda}_{m_t}$ --- ковариационная матрица шума на неисправных сенсорах в момент времени $t$;
\item 
 $\mathbf{\Lambda}_{o_t}$ --- ковариационная матрица шума на исправных сенсорах в момент времени $t$.
\end{itemize}
Составим ECM-алгоритм (Expectation Conditional Maximization алгоритм) для двух случаев:
\begin{itemize}
\item
Известный шум;
\item
Неизвестный шум.
\end{itemize}
\clearpage
\begin{center}
\fontsize{16}{20}\selectfont \color{teal}{\textbf{\S2  Известный шум}}
\end{center}
Воспользуемся ЕCМ-алгоритмом для того, чтобы определить значения параметров $\Psi = (\theta, S)$, пропущенные значения $X_m=\{X_{m_t}\}_{t=1}^G$ рассматриваются как латентные переменные. Наблюдения $X_t$, $t=\overline{1,L}$ предполагаются независимыми и одинаково распределенными.
\begin{center}
\fontsize{14}{18}\selectfont \color{red}{\textbf{Инициализация параметров}}
\end{center}
Оценим вектор углов $\theta^{(0)}$ следующим образом:
\begin{enumerate}
\item
Выберем число $\nu$, которое будет соответствовать первому компоненту вектора $\theta^{(0)}$:
$$
\nu \sim \mathcal{U}([-\pi;\pi]);
$$
\item
Оценим компоненты вектора $\theta^{(0)}$ так:  $\theta^{(0)}_i = \nu + (i-1)\cdot \frac{2\pi}{M}, i = \overline{1,M}$.
\end{enumerate}
Теперь следует составить первоначальную оценку детерминированных сигналов. Предположим, что детерминированные сигналы мы можем представить в форме:
\begin{equation}
S(t) = Ae^{j\omega \cdot t + \phi},
\end{equation}
где $A$ -- амплитуда, $\omega$ -- частота, $\phi$ -- фаза.
Предполагаем, что амплитуда, частота и фаза у сигналов от различных источников могут отличаться, генерируем эти параметры из равномерного распределения. На основе их строим оценку сигналов.
\begin{center}
\fontsize{14}{18}\selectfont \color{red}{\textbf{Е-шаг}}
\end{center}
Требуется найти условное математическое ожидание с учетом апостериорного распределения ненаблюдаемых/пропущенных принятых сигналов и текущей оценки параметров
\begin{equation}
 \Expect_{X_m|X_o, \Psi^{(\tau-1)}}[\log P(X_o, X_m)]
\end{equation}
Сначала найдем апостериорное распределение $P(X_m|X_o=x_o,\Psi)$, воспользуемся формулой Байеса:
\begin{gather}
P(X_m|X_o=x_o,\Psi) = \frac{P(X_o, X_m|\Psi)}{P(X_o|\Psi)} = \frac{P(X|\Psi)}{P(X_o|\Psi)}
\end{gather}
\begin{gather*}
X_t = AS_t + N_t \\
X_t \sim CN(A S_t,\mathbf{\Lambda})\\
X_{o_t} \sim CN( A_{o_t}S_t, \mathbf{\Lambda_{o_t}})\\
\end{gather*}
\begin{gather}
P(X|\Psi) = \prod_{t=1}^G \frac{1}{\pi^L \Det(\mathbf{\Lambda})}e^{-(X_t-AS_t)^H (\mathbf{\Lambda})^{-1}(X_t-AS_t)},
\end{gather}
\begin{gather}
P(X_o|\Psi) = \prod_{t=1}^G \frac{1}{\pi^{L_{o_t}} \Det(\mathbf{\Lambda}_{o_t})}e^{-(X_{o_t}-A_{o_t}S_t)^H (\mathbf{\Lambda}_{o_t})^{-1}(X_{o_t}-A_{o_t}S_t)},
\end{gather}
Параметры апостериорного распределения $P(X_m|X_o=x_o,\Psi)$ можно найти исходя из следующих формул:
\begin{equation}
\left\{ \begin{aligned}
\mu_{x_{m_t}|x_{o_t}} &= \mu_{x_{m_t}} + \Sigma_{x_{m_t},x_{o_t}}\Sigma_{x_{o_t},x_{o_t}}^{-1}\cdot(x_{o_t}-\mu_{x_{o_t}}) \\
\Sigma_{x_{m_t}|x_{o_t}} &= \Sigma_{x_{m_t},x_{m_t}}-\Sigma_{x_{m_t},x_{o_t}}\Sigma_{x_{o_t},x_{o_t}}^{-1}\Sigma_{x_{o_t},x_{m_t}}
\end{aligned} \right.
\end{equation}
В рамках данной задачи:
\begin{equation}
\left\{ \begin{aligned} 
\Sigma_{x_{o_t},x_{o_t}} &= \mathbf{\Lambda}_{o_t} \\
\Sigma_{x_{o_t},x_{m_t}} &= \hat{\Sigma}_{x_{o_t},x_{m_t}} \\
\Sigma_{x_{m_t},x_{o_t}} &= \hat{\Sigma}^H_{x_{o_t},x_{m_t}} \\
\Sigma_{x_{m_t},x_{m_t}} &= \mathbf{\Lambda}_{m_t} \\
\mu_{x_{o_t}} &= A_{o_t}S_t \\
\mu_{x_{m_t}} &= A_{m_t}S_t
\end{aligned} \right.
\end{equation}
где $\hat{\Sigma}_{x_{o_t},x_{m_t}}$ -- выборочная оценка ковариации на основе полных наблюдений, т.е. таких, что $x_t = x_{o_t}$
\begin{equation}
\left\{ \begin{aligned} 
\mu_{x_{m_t}|x_{o_t}} &= A_{m_t}S_t + \hat{\Sigma}^H_{x_{o_t},x_{m_t}}(\mathbf{\Lambda}_{o_t})^{-1}\cdot(x_o-A_{o_t}S_t) \\
\Sigma_{x_{m_t}|x_{o_t}} &= \mathbf{\Lambda}_{m_t}-\hat{\Sigma}^H_{x_{o_t},x_{m_t}}(\mathbf{\Lambda}_{o_t})^{-1}\hat{\Sigma}_{x_{o_t},x_{m_t}}
\end{aligned} \right.
\end{equation}
Оцениваем пропущенные значения для каждого наблюдения через условное математическое ожидание:
$\widetilde{x}_{m_t} = \mu_{x_{m_t}|x_{o_t}}$.
Вернемся к ранее рассмотренному условному математическому ожиданию:
\begin{equation*}
 \Expect_{X_m|X_o, \Psi^{(\tau)}}[\log P(X_o, X_m)].
\end{equation*}
Его следует максимизировать, мы можем перейти от логарифма произведения к сумме логарифмов. Определим, как будет выглядеть это УМО для произвольно выбранного элемента выборки $X_t$:
\begin{equation*}
\begin{gathered}
 \Expect_{X_m|X_o=x_o, \Psi^{(\tau)}}[\log P(X_o, X_m)] = \\
 \sum_{t=1}^G\Expect_{X_{m_t}|X_{o_t}=x_{o_t}, \Psi^{(\tau)}}[\log P(X_{o_t}, X_{m_t})]
\end{gathered}
\end{equation*}
\\
\\
\begin{equation*}
\begin{gathered}
\Expect_{X_{m_t}|X_{o_t}=x_{o_t}, \Psi^{(\tau-1)}}[\log P(X_{o_t}, X_{m_t})]=\\
\Expect_{\Psi^{(\tau-1)}}\Big[\log P\Big(X_{o_t}, X_{m_t}\Big|X_{m_t}|X_{o_t}=x_{o_t}\Big)\Big]=\\
\Expect_{\Psi^{(\tau-1)}}\bigg[\log \left(\frac{1}{\pi^{L}\Det(\Sigma)}e^{-(X_t-\mu)^H\Sigma^{-1}(X_t-\mu)}\right)\Big|X_{m_t}|X_{o_t}=x_{o_t}\bigg]=\\
\Expect_{\Psi^{(\tau-1)}}\bigg[-L \log(\pi) - \log (\Det(\mathbf{\Lambda})) - (X_t - \mu)^H\mathbf{\Lambda}^{-1}(X_t - \mu)\Big|X_{m_t}|X_{o_t}=x_{o_t}\bigg]=\\
-L \log(\pi) - \log (\Det(\mathbf{\Lambda})) + \\ + \Expect_{\Psi^{(\tau-1)}}\bigg[- (X_t - \mu)^H\mathbf{\Lambda}^{-1}(X_t - \mu)\Big|X_{m_t}|X_{o_t}=x_{o_t}\bigg]=\\
-L \log(\pi) - \log (\Det(\mathbf{\Lambda}))  + \\ + \Expect_{\Psi^{(\tau-1)}}\bigg[- 
\begin{bmatrix} X_{m_t} - \mu_{X_{m_t}} \\  X_{o_t} - \mu_{X_{o_t}} \end{bmatrix}^H \mathbf{\Lambda}^{-1}
\begin{bmatrix} X_{m_t} - \mu_{X_{m_t}} \\  X_{o_t} - \mu_{X_{o_t}} \end{bmatrix}\Big|X_{m_t}|X_{o_t}=x_{o_t}\bigg] = \\
-L \log(\pi) - \log (\Det(\mathbf{\Lambda}))  + \\ + \Expect_{\Psi^{(\tau-1)}}\bigg[- 
\begin{bmatrix} X_{m_t} - A_{m_t}S_t \\  X_{o_t} - A_{o_t}S_t  \end{bmatrix}^H
\begin{bmatrix} \mathbf{\Lambda}_{(m_t, m_t)} & \mathbf{O} \\  \mathbf{O} & \mathbf{\Lambda}_{(o_t, o_t)} \end{bmatrix}^{-1}
\begin{bmatrix} X_{m_t} - A_{m_t}S_t \\  X_{o_t} - A_{o_t}S_t  \end{bmatrix}\Big|X_{m_t}|X_{o_t}=x_{o_t}\bigg] = \\
-L \log(\pi) - \log (\Det(\mathbf{\Lambda}))  - \\ \Expect_{\Psi^{(\tau-1)}}
\bigg[ (X_{m_t} - A_{m_t}S_t)^H (\mathbf{\Lambda}^{-1})_{(m_t, m_t)} (X_{m_t} - A_{m_t}S_t) + \\
(X_{o_t} - A_{o_t}S_t)^H (\mathbf{\Lambda}^{-1})_{(o_t, o_t)} (X_{o_t} - A_{o_t}S_t)
\Big|X_{m_t}|X_{o_t}=x_{o_t}\bigg] = \\
\end{gathered}
\end{equation*}
\begin{equation*}
\begin{gathered}
-L \log(\pi) - \log (\Det(\mathbf{\Lambda}))  - (X_{o_t} - A_{o_t}S_t)^H (\mathbf{\Lambda}^{-1})_{(o_t, o_t)} (X_{o_t} - A_{o_t}S_t) - \\ \Expect_{\Psi^{(\tau-1)}} \bigg[ (X_{m_t} - A_{m_t}S_t)^H (\mathbf{\Lambda}^{-1})_{(m_t, m_t)} (X_{m_t} - A_{m_t}S_t) 
\Big|X_{m_t}|X_{o_t}=x_{o_t}\bigg] = \\
\end{gathered}
\end{equation*}
Заметим, что: 
\begin{equation*}
[(\mathbf{\Lambda}^{-1})_{(m_t, m_t)}]^{\frac{1}{2}}X_{m_t} \sim N\Big([(\mathbf{\Lambda}^{-1})_{(m_t, m_t)}]^{\frac{1}{2}}(\widetilde{x}_{m_t}^{(\tau)} - \mu_{x_{m,t}}),[(\mathbf{\Lambda}^{-1})_{(m_t, m_t)}]^{\frac{1}{2}}\mathbf{\Lambda}_{m_t}[(\mathbf{\Lambda}^{-1})_{(m_t, m_t)}]^{\frac{1}{2}}\Big)
\end{equation*}
Выполняем последующие преобразования, используя тот факт, что: $\Expect(WW^H)=\Cov(W,W)+\Expect(W)\Expect(W)^H$.
\begin{equation*}
\begin{gathered}
\Expect_{\Psi^{(\tau-1)}}\bigg[X_{m_t}^H (\mathbf{\Lambda}^{-1})_{(m_t, m_t)}X_{m_t}   \Big|X_{m_t}|X_{o_t}=x_{o_t}\bigg] = \\
\Expect_{X_{m_t}|X_{o_t}=x_{o_t}, \Psi^{(\tau-1)}}\bigg[\Big[[(\mathbf{\Lambda}^{-1})_{(m_t, m_t)}]^{\frac{1}{2}}X_{m_t}\Big]^H\Big[[(\mathbf{\Lambda}^{-1})_{(m_t, m_t)}]^{\frac{1}{2}}X_{m_t}\Big]\bigg] = \\
\Expect_{X_{m_t}|X_{o_t}=x_{o_t}, \Psi^{(\tau-1)}}[W_t^HW_t] = \\
\Tr \Big( \Expect_{\Psi^{(\tau-1)},X_{m_t}|X_{o_t}=x_{o_t}}[W_tW_t^H] \Big) = \\
\Tr \Big( \big[(\mathbf{\Lambda}^{-1})_{(m_t, m_t)}\big]^{\frac{1}{2}}\mathbf{\Lambda}_{m_t}\big[(\mathbf{\Lambda}^{-1})_{(m_t, m_t)}\big]^{\frac{1}{2}} \Big) + \\ \Tr \Big( \big[[(\mathbf{\Lambda}^{-1})_{(m_t, m_t)}]^{\frac{1}{2}}(\widetilde{x}_{m_t}^{(\tau)} - \mu_{X_{m_t}})\big]\big[[(\mathbf{\Lambda}^{-1})_{(m_t, m_t)}\big]^{\frac{1}{2}}(\widetilde{x}_{m_t}^{(\tau)} - \mu_{X_{m_t}})]^H \Big)= \\
\Tr \Big( \big[(\mathbf{\Lambda}^{-1})_{(m_t, m_t)}\big]\mathbf{\Lambda}_{m_t} \Big) + \Tr \Big((\widetilde{x}_{m_t}^{(\tau)} - \mu_{X_{m_t}})^H\big[(\mathbf{\Lambda}^{-1})_{(m_t, m_t)}\big](\widetilde{x}_{m_t}^{(\tau)} - \mu_{X_{m_t}}) \Big) \\
\end{gathered}
\end{equation*}
\begin{equation*}
\begin{gathered}
\Expect_{X_{m_t}|X_{o_t}=x_{o_t}, \Psi^{(\tau-1)}}[\log P(X_{o_t}, X_{m_t})]=\\
-L \log(\pi) - \log (\Det(\mathbf{\Lambda})) - (x_{o_t}-\mu_{X_{o_t}})^H (\mathbf{\Lambda}^{-1})_{(o_t, o_t)}(x_{o_t}-\mu_{X_{o_t}}) -  \\ \Tr \Big( [(\mathbf{\Lambda}^{-1})_{(m_t, m_t)}]\mathbf{\Lambda}_{m_t} \Big) - \Tr \Big((\widetilde{x}_{m_t}^{(\tau)} - \mu_{X_{m_t}})^H[(\mathbf{\Lambda}^{-1})_{(m_t, m_t)}](\widetilde{x}_{m_t}^{(\tau)} - \mu_{X_{m_t}}) \Big) 
\end{gathered}
\end{equation*}
Таким образом, оптимизируемая функция (с учетом исключения слагаемых, независящих от параметров) примет следующий вид:
\begin{equation*}
\begin{gathered}
 \Expect_{X_m|X_o=x_o, \Psi^{(\tau-1)}}[\log P(X_o, X_m)] = \\
 \sum_{t=1}^G\Expect_{X_{m_t}|X_{o_t}=x_{o_t}, \Psi^{(\tau)}}[\log P(X_{o_t}, X_{m_t})] = \\
\sum_{t=1}^G \bigg[ -L \log(\pi) - \log (\Det(\mathbf{\Lambda})) - (x_{o_t}-\mu_{x_{o_t}}) ^H (\mathbf{\Lambda}^{-1})_{(o_t, o_t)}(x_{o_t}-\mu_{X_{o_t}})  -  \\ \Tr \Big( [(\mathbf{\Lambda}^{-1})_{(m_t, m_t)}]\mathbf{\Lambda}_{m_t} \Big) - \Tr \Big((\widetilde{x}_{m_t}^{(\tau)} - \mu_{X_{m_t}})^H[(\mathbf{\Lambda}^{-1})_{(m_t, m_t)}](\widetilde{x}_{m_t}^{(\tau)} - \mu_{X_{m_t}}) \Big)  \bigg] =\\
\sum_{t=1}^G \bigg[- (x_{o_t}-\mu_{x_{o_t}}) ^H (\mathbf{\Lambda}^{-1})_{(o_t, o_t)}(x_{o_t}-\mu_{X_{o_t}})  - \Tr \Big((\widetilde{x}_{m_t}^{(\tau)} - \mu_{X_{m_t}})^H[(\mathbf{\Lambda}^{-1})_{(m_t, m_t)}](\widetilde{x}_{m_t}^{(\tau)} - \mu_{X_{m_t}}) \Big)  \bigg]
\end{gathered}
\end{equation*}
Рассмотрим второе слагаемое: внутри следа находится скаляр, соответственно, преобразуем выражение:
\begin{equation*}
\begin{gathered}
\sum_{t=1}^G \bigg[- (x_{o_t}-\mu_{x_{o_t}}) ^H (\mathbf{\Lambda}^{-1})_{(o_t, o_t)}(x_{o_t}-\mu_{X_{o_t}})  - \Tr \Big((\widetilde{x}_{m_t}^{(\tau)} - \mu_{X_{m_t}})^H[(\mathbf{\Lambda}^{-1})_{(m_t, m_t)}](\widetilde{x}_{m_t}^{(\tau)} - \mu_{X_{m_t}}) \Big)  \bigg] = \\
\sum_{t=1}^G \bigg[- (x_{o_t}-\mu_{x_{o_t}}) ^H (\mathbf{\Lambda}^{-1})_{(o_t, o_t)}(x_{o_t}-\mu_{X_{o_t}})  - (\widetilde{x}_{m_t}^{(\tau)} - \mu_{X_{m_t}})^H[(\mathbf{\Lambda}^{-1})_{(m_t, m_t)}](\widetilde{x}_{m_t}^{(\tau)} - \mu_{X_{m_t}})  \bigg] = \\
= \sum_{t=1}^G \Big( - (X_t-AS_t)^H \mathbf{\Lambda}^{-1}(X_t-AS_t) \Big) = \\
- ||\mathbf{\Lambda}^{-1/2}(X-AS)||_F^2
\end{gathered}
\end{equation*}
\clearpage
\begin{center}
\fontsize{14}{18}\selectfont \color{red}{\textbf{M-шаг}}
\end{center}
Требуется найти наилучшую оценку параметров, решив следующую задачу оптимизации:
\begin{equation*}
\begin{gathered}
\Psi^{(\tau)}=\argmax_{\Psi} \Expect_{X_m|X_o=x_o, \Psi^{(\tau-1)}}[\log P(X_o, X_m)] =\\
\argmax_{\Psi}  \sum_{t=1}^G \bigg[ -L \log(\pi) - \log (\Det(\mathbf{\Lambda})) - (x_{o_t}-\mu_{X_{o_t}})^H (\mathbf{\Lambda}^{-1})_{(o_t, o_t)}(x_{o_t}-\mu_{X_{o_t}}) -  \\ \Tr \Big( [(\mathbf{\Lambda}^{-1})_{(m_t, m_t)}]\mathbf{\Lambda}_{m_t} \Big) - \Tr \Big((\widetilde{x}_{m_t}^{(\tau)} - \mu_{X_{m_t}})^H[(\mathbf{\Lambda}^{-1})_{(m_t, m_t)}](\widetilde{x}_{m_t}^{(\tau)} - \mu_{X_{m_t}}) \Big)  \bigg] = \\
\argmin_{\Psi} ||\mathbf{\Lambda}^{-1/2}(X-AS)||_F^2
\end{gathered}
\end{equation*}
\begin{center}
\fontsize{14}{18}\selectfont \color{red}{\textbf{Первый СM-шаг}}
\end{center}
Оценим углы прибытия сигналов $\theta$, но оставляем оценку сигналов $S$ фиксированной: $S = S^{(\tau-1)}$.
\begin{equation*}
\begin{gathered}
\theta^{(\tau)}= \argmax_{\theta} Q(\theta | \theta^{(\tau-1)}) = \\
\argmin_{\theta} ||\mathbf{\Lambda}^{-1/2}(X-AS)||_F^2
\end{gathered}
\end{equation*}
\\
\begin{center}
\fontsize{14}{18}\selectfont \color{red}{\textbf{Второй СM-шаг}}
\end{center}
Оценим сигналы $S$, но оставляем оценку углов прибытия сигналов $\theta$ фиксированной: $\theta = \theta^{(\tau)}$
Можем теперь численно решить следующую систему уравнений относительно $\mathbf{S}^{(\tau)}=\{S_t^{(\tau)}\}_{t=1}^G$:
\begin{equation}
S^{\tau} = \argmin_S(X-AS)^HQ^{-1}(X-AS)
\end{equation}
\begin{equation}
\left\{
\begin{aligned}
S_1^{(\tau)} &= (A(\theta^{(\tau)})^H Q^{-1} A(\theta^{(\tau)}))^{-1}A(\theta^{(\tau)})^H Q^{-1}  \widetilde{x}_1^{(\tau)} \\
S_2^{(\tau)} &= (A(\theta^{(\tau)})^H Q^{-1} A(\theta^{(\tau)}))^{-1}A(\theta^{(\tau)})^H Q^{-1}  \widetilde{x}_2^{(\tau)} \\
&\vdots \\
S_G^{(\tau)} &= (A(\theta^{(\tau)})^H Q^{-1} A(\theta^{(\tau)}))^{-1}A(\theta^{(\tau)})^H Q^{-1}  \widetilde{x}_G^{(\tau)}
\end{aligned}
\right.
\end{equation}
Шаги повторяются либо до достижения максимального числа итераций либо до сходимости оценок параметров, полученных на соседних двух итерациях.
\clearpage
\begin{center}
\fontsize{16}{20}\selectfont \color{teal}{\textbf{\S3 Неизвестный шум}}
\end{center}
Воспользуемся ЕCМ-алгоритмом для того, чтобы определить значения параметров $\Psi = (\theta, S, \mathbf{\Lambda})$, пропущенные значения $X_m=\{X_{m_t}\}_{t=1}^G$ рассматриваются как латентные переменные. Наблюдения $X_t$, $t=\overline{1,L}$ предполагаются независимыми и одинаково распределенными.
\begin{center}
\fontsize{14}{18}\selectfont \color{red}{\textbf{Инициализация параметров}}
\end{center}
Оценим вектор углов $\theta^{(0)}$ следующим образом:
\begin{enumerate}
\item
Выберем число $\nu$, которое будет соответствовать первому компоненту вектора $\theta^{(0)}$:
$$
\nu \sim \mathcal{U}([-\pi;\pi]);
$$
\item
Оценим компоненты вектора $\theta^{(0)}$ так:  $\theta^{(0)}_i = \nu + (i-1)\cdot \frac{2\pi}{M}, i = \overline{1,M}$.
\end{enumerate}
Теперь следует составить первоначальную оценку детерминированных сигналов. Предположим, что детерминированные сигналы мы можем представить в форме:
\begin{equation}
S(t) = Ae^{j\omega \cdot t + \phi},
\end{equation}
где $A$ -- амплитуда, $\omega$ -- частота, $\phi$ -- фаза.
Предполагаем, что амплитуда, частота и фаза у сигналов от различных источников могут отличаться, генерируем эти параметры из равномерного распределения. На основе их строим оценку сигналов.
Шумы на сенсорах задаются  с учетом дисперсии остатков
\begin{equation}
\lambda^{(0)}_{jj} = \frac{1}{G} \sum_{t=1}^G |X_{j,t} - (AS)_{j,t}|^2
\end{equation}.
\begin{center}
\fontsize{14}{18}\selectfont \color{red}{\textbf{Е-шаг}}
\end{center}
Требуется найти условное математическое ожидание с учетом апостериорного распределения ненаблюдаемых/пропущенных принятых сигналов и текущей оценки параметров
\begin{equation}
 \Expect_{X_m|X_o, \Psi^{(\tau-1)}}[\log P(X_o, X_m)]
\end{equation}
Сначала найдем апостериорное распределение $P(X_m|X_o=x_o,\Psi)$, воспользуемся формулой Байеса:
\begin{gather}
P(X_m|X_o=x_o,\Psi) = \frac{P(X_o, X_m|\Psi)}{P(X_o|\Psi)} = \frac{P(X|\Psi)}{P(X_o|\Psi)}
\end{gather}
\begin{gather*}
X_t = AS_t + N_t \\
X_t \sim CN(A S_t,\mathbf{\Lambda})\\
X_{o_t} \sim CN( A_{o_t}S_t, \mathbf{\Lambda_{o_t}})\\
\end{gather*}
\begin{gather}
P(X|\Psi) = \prod_{t=1}^G \frac{1}{\pi^L \Det(\mathbf{\Lambda})}e^{-(X_t-AS_t)^H (\mathbf{\Lambda})^{-1}(X_t-AS_t)},
\end{gather}
\begin{gather}
P(X_o|\Psi) = \prod_{t=1}^G \frac{1}{\pi^{L_{o_t}} \Det(\mathbf{\Lambda}_{o_t})}e^{-(X_{o_t}-A_{o_t}S_t)^H (\mathbf{\Lambda}_{o_t})^{-1}(X_{o_t}-A_{o_t}S_t)},
\end{gather}
Параметры апостериорного распределения $P(X_m|X_o=x_o,\Psi)$ можно найти исходя из следующих формул:
\begin{equation}
\left\{ \begin{aligned}
\mu_{x_{m_t}|x_{o_t}} &= \mu_{x_{m_t}} + \Sigma_{x_{m_t},x_{o_t}}\Sigma_{x_{o_t},x_{o_t}}^{-1}\cdot(x_{o_t}-\mu_{x_{o_t}}) \\
\Sigma_{x_{m_t}|x_{o_t}} &= \Sigma_{x_{m_t},x_{m_t}}-\Sigma_{x_{m_t},x_{o_t}}\Sigma_{x_{o_t},x_{o_t}}^{-1}\Sigma_{x_{o_t},x_{m_t}}
\end{aligned} \right.
\end{equation}
В рамках данной задачи:
\begin{equation}
\left\{ \begin{aligned} 
\Sigma_{x_{o_t},x_{o_t}} &= \mathbf{\Lambda}_{o_t} \\
\Sigma_{x_{o_t},x_{m_t}} &= \hat{\Sigma}_{x_{o_t},x_{m_t}} \\
\Sigma_{x_{m_t},x_{o_t}} &= \hat{\Sigma}^H_{x_{o_t},x_{m_t}} \\
\Sigma_{x_{m_t},x_{m_t}} &= \mathbf{\Lambda}_{m_t} \\
\mu_{x_{o_t}} &= A_{o_t}S_t \\
\mu_{x_{m_t}} &= A_{m_t}S_t
\end{aligned} \right.
\end{equation}
где $\hat{\Sigma}_{x_{o_t},x_{m_t}}$ -- выборочная оценка ковариации на основе полных наблюдений, т.е. таких, что $x_t = x_{o_t}$
\begin{equation}
\left\{ \begin{aligned} 
\mu_{x_{m_t}|x_{o_t}} &= A_{m_t}S_t + \hat{\Sigma}^H_{x_{o_t},x_{m_t}}(\mathbf{\Lambda}_{o_t})^{-1}\cdot(x_o-A_{o_t}S_t) \\
\Sigma_{x_{m_t}|x_{o_t}} &= \mathbf{\Lambda}_{m_t}-\hat{\Sigma}^H_{x_{o_t},x_{m_t}}(\mathbf{\Lambda}_{o_t})^{-1}\hat{\Sigma}_{x_{o_t},x_{m_t}}
\end{aligned} \right.
\end{equation}
Оцениваем пропущенные значения для каждого наблюдения через условное математическое ожидание:
$\widetilde{x}_{m_t} = \mu_{x_{m_t}|x_{o_t}}$.
Вернемся к ранее рассмотренному условному математическому ожиданию:
\begin{equation*}
 \Expect_{X_m|X_o, \Psi^{(\tau)}}[\log P(X_o, X_m)].
\end{equation*}
Его следует максимизировать, мы можем перейти от логарифма произведения к сумме логарифмов. Определим, как будет выглядеть это УМО для произвольно выбранного элемента выборки $X_t$:
\begin{equation*}
\begin{gathered}
 \Expect_{X_m|X_o=x_o, \Psi^{(\tau)}}[\log P(X_o, X_m)] = \\
 \sum_{t=1}^G\Expect_{X_{m_t}|X_{o_t}=x_{o_t}, \Psi^{(\tau)}}[\log P(X_{o_t}, X_{m_t})]
\end{gathered}
\end{equation*}
\\
\\
\begin{equation*}
\begin{gathered}
\Expect_{X_{m_t}|X_{o_t}=x_{o_t}, \Psi^{(\tau-1)}}[\log P(X_{o_t}, X_{m_t})]=\\
\Expect_{\Psi^{(\tau-1)}}\Big[\log P\Big(X_{o_t}, X_{m_t}\Big|X_{m_t}|X_{o_t}=x_{o_t}\Big)\Big]=\\
\Expect_{\Psi^{(\tau-1)}}\bigg[\log \left(\frac{1}{\pi^{L}\Det(\Sigma)}e^{-(X_t-\mu)^H\Sigma^{-1}(X_t-\mu)}\right)\Big|X_{m_t}|X_{o_t}=x_{o_t}\bigg]=\\
\Expect_{\Psi^{(\tau-1)}}\bigg[-L \log(\pi) - \log (\Det(\mathbf{\Lambda})) - (X_t - \mu)^H\mathbf{\Lambda}^{-1}(X_t - \mu)\Big|X_{m_t}|X_{o_t}=x_{o_t}\bigg]=\\
-L \log(\pi) - \log (\Det(\mathbf{\Lambda})) + \\ + \Expect_{\Psi^{(\tau-1)}}\bigg[- (X_t - \mu)^H\mathbf{\Lambda}^{-1}(X_t - \mu)\Big|X_{m_t}|X_{o_t}=x_{o_t}\bigg]=\\
-L \log(\pi) - \log (\Det(\mathbf{\Lambda}))  + \\ + \Expect_{\Psi^{(\tau-1)}}\bigg[- 
\begin{bmatrix} X_{m_t} - \mu_{X_{m_t}} \\  X_{o_t} - \mu_{X_{o_t}} \end{bmatrix}^H \mathbf{\Lambda}^{-1}
\begin{bmatrix} X_{m_t} - \mu_{X_{m_t}} \\  X_{o_t} - \mu_{X_{o_t}} \end{bmatrix}\Big|X_{m_t}|X_{o_t}=x_{o_t}\bigg] = \\
-L \log(\pi) - \log (\Det(\mathbf{\Lambda}))  + \\ + \Expect_{\Psi^{(\tau-1)}}\bigg[- 
\begin{bmatrix} X_{m_t} - A_{m_t}S_t \\  X_{o_t} - A_{o_t}S_t  \end{bmatrix}^H
\begin{bmatrix} \mathbf{\Lambda}_{(m_t, m_t)} & \mathbf{O} \\  \mathbf{O} & \mathbf{\Lambda}_{(o_t, o_t)} \end{bmatrix}^{-1}
\begin{bmatrix} X_{m_t} - A_{m_t}S_t \\  X_{o_t} - A_{o_t}S_t  \end{bmatrix}\Big|X_{m_t}|X_{o_t}=x_{o_t}\bigg] = \\
-L \log(\pi) - \log (\Det(\mathbf{\Lambda}))  - \\ \Expect_{\Psi^{(\tau-1)}}
\bigg[ (X_{m_t} - A_{m_t}S_t)^H (\mathbf{\Lambda}^{-1})_{(m_t, m_t)} (X_{m_t} - A_{m_t}S_t) + \\
(X_{o_t} - A_{o_t}S_t)^H (\mathbf{\Lambda}^{-1})_{(o_t, o_t)} (X_{o_t} - A_{o_t}S_t)
\Big|X_{m_t}|X_{o_t}=x_{o_t}\bigg] = \\
\end{gathered}
\end{equation*}
\begin{equation*}
\begin{gathered}
-L \log(\pi) - \log (\Det(\mathbf{\Lambda}))  - (X_{o_t} - A_{o_t}S_t)^H (\mathbf{\Lambda}^{-1})_{(o_t, o_t)} (X_{o_t} - A_{o_t}S_t) - \\ \Expect_{\Psi^{(\tau-1)}} \bigg[ (X_{m_t} - A_{m_t}S_t)^H (\mathbf{\Lambda}^{-1})_{(m_t, m_t)} (X_{m_t} - A_{m_t}S_t) 
\Big|X_{m_t}|X_{o_t}=x_{o_t}\bigg] = \\
\end{gathered}
\end{equation*}
Заметим, что: 
\begin{equation*}
[(\mathbf{\Lambda}^{-1})_{(m_t, m_t)}]^{\frac{1}{2}}X_{m_t} \sim N\Big([(\mathbf{\Lambda}^{-1})_{(m_t, m_t)}]^{\frac{1}{2}}(\widetilde{x}_{m_t}^{(\tau)} - \mu_{x_{m,t}}),[(\mathbf{\Lambda}^{-1})_{(m_t, m_t)}]^{\frac{1}{2}}\mathbf{\Lambda}_{m_t}[(\mathbf{\Lambda}^{-1})_{(m_t, m_t)}]^{\frac{1}{2}}\Big)
\end{equation*}
Выполняем последующие преобразования, используя тот факт, что: $\Expect(WW^H)=\Cov(W,W)+\Expect(W)\Expect(W)^H$.
\begin{equation*}
\begin{gathered}
\Expect_{\Psi^{(\tau-1)}}\bigg[X_{m_t}^H (\mathbf{\Lambda}^{-1})_{(m_t, m_t)}X_{m_t}   \Big|X_{m_t}|X_{o_t}=x_{o_t}\bigg] = \\
\Expect_{X_{m_t}|X_{o_t}=x_{o_t}, \Psi^{(\tau-1)}}\bigg[\Big[[(\mathbf{\Lambda}^{-1})_{(m_t, m_t)}]^{\frac{1}{2}}X_{m_t}\Big]^H\Big[[(\mathbf{\Lambda}^{-1})_{(m_t, m_t)}]^{\frac{1}{2}}X_{m_t}\Big]\bigg] = \\
\Expect_{X_{m_t}|X_{o_t}=x_{o_t}, \Psi^{(\tau-1)}}[W_t^HW_t] = \\
\Tr \Big( \Expect_{X_{m_t}|X_{o_t}=x_{o_t}, \Psi^{(\tau-1)}}[W_tW_t^H] \Big) = \\
\Tr \Big( \big[(\mathbf{\Lambda}^{-1})_{(m_t, m_t)}\big]^{\frac{1}{2}}\mathbf{\Lambda}_{m_t}\big[(\mathbf{\Lambda}^{-1})_{(m_t, m_t)}\big]^{\frac{1}{2}} \Big) + \\ \Tr \Big( \big[[(\mathbf{\Lambda}^{-1})_{(m_t, m_t)}]^{\frac{1}{2}}(\widetilde{x}_{m_t}^{(\tau)} - \mu_{X_{m_t}})\big]\big[[(\mathbf{\Lambda}^{-1})_{(m_t, m_t)}\big]^{\frac{1}{2}}(\widetilde{x}_{m_t}^{(\tau)} - \mu_{X_{m_t}})]^H \Big)= \\
\Tr \Big( \big[(\mathbf{\Lambda}^{-1})_{(m_t, m_t)}\big]\mathbf{\Lambda}_{m_t} \Big) + \Tr \Big((\widetilde{x}_{m_t}^{(\tau)} - \mu_{X_{m_t}})^H\big[(\mathbf{\Lambda}^{-1})_{(m_t, m_t)}\big](\widetilde{x}_{m_t}^{(\tau)} - \mu_{X_{m_t}}) \Big) \\
\end{gathered}
\end{equation*}
\begin{equation*}
\begin{gathered}
\Expect_{X_{m_t}|X_{o_t}=x_{o_t}, \Psi^{(\tau-1)}}[\log P(X_{o_t}, X_{m_t})]=\\
-L \log(\pi) - \log (\Det(\mathbf{\Lambda})) - (x_{o_t}-\mu_{X_{o_t}})^H (\mathbf{\Lambda}^{-1})_{(o_t, o_t)}(x_{o_t}-\mu_{X_{o_t}}) -  \\ \Tr \Big( [(\mathbf{\Lambda}^{-1})_{(m_t, m_t)}]\mathbf{\Lambda}_{m_t} \Big) - \Tr \Big((\widetilde{x}_{m_t}^{(\tau)} - \mu_{X_{m_t}})^H[(\mathbf{\Lambda}^{-1})_{(m_t, m_t)}](\widetilde{x}_{m_t}^{(\tau)} - \mu_{X_{m_t}}) \Big) 
\end{gathered}
\end{equation*}
Таким образом, оптимизируемая функция (с учетом исключения слагаемых, независящих от параметров) примет следующий вид:
\begin{equation*}
\begin{gathered}
 \Expect_{X_m|X_o=x_o, \Psi^{(\tau-1)}}[\log P(X_o, X_m)] = \\
 \sum_{t=1}^G\Expect_{X_{m_t}|X_{o_t}=x_{o_t}, \Psi^{(\tau)}}[\log P(X_{o_t}, X_{m_t})] = \\
\sum_{t=1}^G \bigg[ -L \log(\pi) - \log (\Det(\mathbf{\Lambda})) - (x_{o_t}-\mu_{x_{o_t}}) ^H (\mathbf{\Lambda}^{-1})_{(o_t, o_t)}(x_{o_t}-\mu_{X_{o_t}})  -  \\ \Tr \Big( [(\mathbf{\Lambda}^{-1})_{(m_t, m_t)}]\mathbf{\Lambda}_{m_t} \Big) - \Tr \Big((\widetilde{x}_{m_t}^{(\tau)} - \mu_{X_{m_t}})^H[(\mathbf{\Lambda}^{-1})_{(m_t, m_t)}](\widetilde{x}_{m_t}^{(\tau)} - \mu_{X_{m_t}}) \Big)  \bigg] =\\
\sum_{t=1}^G \bigg[- (x_{o_t}-\mu_{x_{o_t}}) ^H (\mathbf{\Lambda}^{-1})_{(o_t, o_t)}(x_{o_t}-\mu_{X_{o_t}})  - \Tr \Big((\widetilde{x}_{m_t}^{(\tau)} - \mu_{X_{m_t}})^H[(\mathbf{\Lambda}^{-1})_{(m_t, m_t)}](\widetilde{x}_{m_t}^{(\tau)} - \mu_{X_{m_t}}) \Big)  \bigg]
\end{gathered}
\end{equation*}
Рассмотрим второе слагаемое: внутри следа находится скаляр, соответственно, преобразуем выражение:
\begin{equation*}
\begin{gathered}
\sum_{t=1}^G \bigg[- (x_{o_t}-\mu_{x_{o_t}}) ^H (\mathbf{\Lambda}^{-1})_{(o_t, o_t)}(x_{o_t}-\mu_{X_{o_t}})  - \Tr \Big((\widetilde{x}_{m_t}^{(\tau)} - \mu_{X_{m_t}})^H[(\mathbf{\Lambda}^{-1})_{(m_t, m_t)}](\widetilde{x}_{m_t}^{(\tau)} - \mu_{X_{m_t}}) \Big)  \bigg] = \\
\sum_{t=1}^G \bigg[- (x_{o_t}-\mu_{x_{o_t}}) ^H (\mathbf{\Lambda}^{-1})_{(o_t, o_t)}(x_{o_t}-\mu_{X_{o_t}})  - (\widetilde{x}_{m_t}^{(\tau)} - \mu_{X_{m_t}})^H[(\mathbf{\Lambda}^{-1})_{(m_t, m_t)}](\widetilde{x}_{m_t}^{(\tau)} - \mu_{X_{m_t}})  \bigg] = \\
= \sum_{t=1}^G \Big( - (X_t-AS_t)^H \mathbf{\Lambda}^{-1}(X_t-AS_t) \Big) = \\
- ||\mathbf{\Lambda}^{-1/2}(X-AS)||_F^2
\end{gathered}
\end{equation*}
\clearpage
\begin{center}
\fontsize{14}{18}\selectfont \color{red}{\textbf{M-шаг}}
\end{center}
Требуется найти наилучшую оценку параметров, решив следующую задачу оптимизации:
\begin{equation*}
\begin{gathered}
\Psi^{(\tau)}=\argmax_{\Psi} \Expect_{X_m|X_o=x_o, \Psi^{(\tau-1)}}[\log P(X_o, X_m)] =\\
\argmax_{\Psi}  \sum_{t=1}^G \bigg[ -L \log(\pi) - \log (\Det(\mathbf{\Lambda})) - (x_{o_t}-\mu_{X_{o_t}})^H (\mathbf{\Lambda}^{-1})_{(o_t, o_t)}(x_{o_t}-\mu_{X_{o_t}}) -  \\ \Tr \Big( [(\mathbf{\Lambda}^{-1})_{(m_t, m_t)}]\mathbf{\Lambda}_{m_t} \Big) - \Tr \Big((\widetilde{x}_{m_t}^{(\tau)} - \mu_{X_{m_t}})^H[(\mathbf{\Lambda}^{-1})_{(m_t, m_t)}](\widetilde{x}_{m_t}^{(\tau)} - \mu_{X_{m_t}}) \Big)  \bigg] = \\
\argmin_{\Psi} ||\mathbf{\Lambda}^{-1/2}(X-AS)||_F^2
\end{gathered}
\end{equation*}
\begin{center}
\fontsize{14}{18}\selectfont \color{red}{\textbf{Первый СM-шаг}}
\end{center}
Оценим углы прибытия сигналов $\theta$, но оставляем оценку сигналов $S$ и оценку ковариации шума $\mathbf{\Lambda}$ фиксироваными: $S = S^{(\tau-1)}, \mathbf{\Lambda}=\mathbf{\Lambda}^{(\tau-1)}$.
\begin{equation*}
\begin{gathered}
\theta^{(\tau)}= \argmax_{\theta} Q(\theta | \theta^{(\tau-1)}) = \\
\argmin_{\theta} ||[\mathbf{\Lambda}^{(\tau-1)}]^{-1/2}(X^{(\tau)}-A^{(\tau-1)}S^{(\tau-1)})||_F^2
\end{gathered}
\end{equation*}
\\
\begin{center}
\fontsize{14}{18}\selectfont \color{red}{\textbf{Второй СM-шаг}}
\end{center}
Оценим сигналы $S$, но оставляем оценку углов прибытия сигналов $\theta$ и оценку ковариации шума $\mathbf{\Lambda}$ фиксироваными: $\theta = \theta^{(\tau)}, \mathbf{\Lambda}=\mathbf{\Lambda}^{(\tau-1)}$. Обозначим через $A^{(\tau)}$ оценку матрицы управляющих векторов $A(\theta^{(\tau)})$ после получения $\theta^{(\tau)}$
Можем теперь численно решить следующую систему уравнений относительно $\mathbf{S}^{(\tau)}=\{S_t^{(\tau)}\}_{t=1}^G$:
\begin{equation}
S^{\tau} = \argmin_S(X^{(\tau)}-A^{(\tau)}S^{(\tau-1)})^H[\mathbf{\Lambda}^{(\tau-1)}]^{-1}(X^{(\tau)}-A^{(\tau)}S^{(\tau-1)})
\end{equation}
\begin{equation}
\left\{
\begin{aligned}
S_1^{(\tau)} &= (A^{(\tau)})^H [\mathbf{\Lambda}^{(\tau-1)}]^{-1} A^{(\tau)})^{-1}(A^{(\tau)})^H [\mathbf{\Lambda}^{(\tau-1)}]^{-1}  \widetilde{x}_1^{(\tau)} \\
S_2^{(\tau)} &= (A^{(\tau)})^H [\mathbf{\Lambda}^{(\tau-1)}]^{-1} A^{(\tau)})^{-1}(A^{(\tau)})^H [\mathbf{\Lambda}^{(\tau-1)}]^{-1}  \widetilde{x}_2^{(\tau)} \\
&\vdots \\
S_G^{(\tau)} &= (A^{(\tau)})^H [\mathbf{\Lambda}^{(\tau-1)}]^{-1} A^{(\tau)})^{-1}(A^{(\tau)})^H [\mathbf{\Lambda}^{(\tau-1)}]^{-1}  \widetilde{x}_G^{(\tau)}
\end{aligned}
\right.
\end{equation}
где $\widetilde{x}_t^{(\tau)}$ --- вектор $x_t$, в котором пропущенные значения $x_{m_t}$ оценены с помощью $\widetilde{x}_{m_t}^{(\tau)}$.
\begin{center}
\fontsize{14}{18}\selectfont \color{red}{\textbf{Третий СM-шаг}}
\end{center}
Оценим  ковариацию шума $\mathbf{\Lambda}$, но оставляем оценку углов прибытия сигналов $\theta$ и оценку сигналов фиксироваными: $\theta = \theta^{(\tau)}, S = S^{(\tau)}$.
Один из вариантов оценить диагональные элементы ковариационной матрицы шума:
\begin{equation}
\lambda^{(\tau)}_{jj} = \frac{1}{G} \sum_{t=1}^G |X^{(\tau)}_{j,t} - (A^{(\tau)}S^{(\tau)})_{j,t}|^2
\end{equation}
Если же напрямую оптимизировать целевую функцию $Q(\theta | \theta^{(\tau-1)})$, можем получить иную аналитическую оценку ковариационной матрицы шума:
Пусть $\widetilde{x}_t^{(\tau)}$ --- вектор $x_t$, в котором пропущенные значения $x_{m_t}$ оценены с помощью $\widetilde{x}_{m_t}^{(\tau)}$.
\begin{align*}
\nabla_{\Lambda^{-1}} Q(\theta \text{ | } \theta^{(\tau-1)}) &= \sum_{t = 1}^{G} \nabla_{\Sigma^{-1}} Q_t (\theta \text{ | } \theta^{(\tau-1)}) \\
&= \sum_{t = 1}^{G} \frac{1}{2} \nabla_{\Sigma^{-1}} \log |\Sigma^{-1}| - \frac{1}{2} \nabla_{\Sigma^{-1}} \Tr \big( (\Sigma^{-1})_{(m_t, m_t)} \Sigma^{(\tau)}_{x_{m_t}|x_{o_t}} \big) \\
&\hspace{10pt} - \sum_{t = 1}^{G} \frac{1}{2} \nabla_{\Sigma^{-1}}  (\widetilde{x}_t^{(\tau)}- A^{(\tau)}S^{(\tau)})^H \Sigma^{-1}  (\widetilde{x}_t^{(\tau)}- A^{(\tau)}S^{(\tau)}) \\
&= \frac{1}{2} \sum_{t = 1}^{G} \nabla_{\Sigma^{-1}} \log |\Sigma^{-1}| - \nabla_{\Sigma^{-1}} \Tr \big( (\Sigma^{-1})_{(m_t, m_t)} \Sigma^{(\tau)}_{x_{m_t}|x_{o_t}} \big) \\
&\hspace{10pt} - \frac{1}{2} \sum_{t = 1}^{G} \nabla_{\Sigma^{-1}} \Tr \big(  (\widetilde{x}_t^{(\tau)}- A^{(\tau)}S^{(\tau)})^H \Sigma^{-1}  (\widetilde{x}_t^{(\tau)}- A^{(\tau)}S^{(\tau)}) \big)
\end{align*}
\begin{align*}
\nabla_{\Lambda^{-1}} Q(\theta \text{ | } \theta^{(\tau-1)}) &= \frac{1}{2} \big[ \sum_{t = 1}^{G} \Lambda -  \widetilde{\Sigma}_{t}^{(\tau)} \big] - \frac{1}{2} \sum_{t = 1}^{G}  (\widetilde{x}_t^{(\tau)}- A^{(\tau)}S^{(\tau)}) (\widetilde{x}_t^{(\tau)}- A^{(\tau)}S^{(\tau)})^H \\
&= \frac{1}{2} \big[ G \Lambda - \sum_{t = 1}^{G} \widetilde{\Sigma}_{t}^{(\tau)} \big] - \frac{1}{2} \sum_{t = 1}^{G}  (\widetilde{x}_t^{(\tau)}- A{(\tau)}S^{(\tau)}) (\widetilde{x}_t^{(\tau)}- A{(\tau)}S^{(\tau)})^H
\end{align*}
где $\Sigma_{t}^{(\tau)}$ --- матрица размера $L \times L$, в которой все элементы являются нулями, за исключением тех, что стоят на пересечении строк с номерами $j_1 \in m_t$ и столбцов с номерами $j_2 \in m_t$: они заменены величиной $\Sigma^{(\tau)}_{x_{m_t}|x_{o_t}}$.
Приравняем производную к нулю, чтобы получить значение $\Sigma$, соответствующее максимуму.
\begin{align*}
O &= \frac{1}{2} \big[ G \Lambda^{(\tau)} - \sum_{t = 1}^{G} \widetilde{\Sigma}_{t}^{(\tau)} \big] - \frac{1}{2} \sum_{t = 1}^{G} (\widetilde{x}_t^{(\tau)}- A^{(\tau)}S^{(\tau)})(\widetilde{x}_t^{(\tau)}- A^{(\tau)}S^{(\tau)})^H \\
&= \big[ G \Lambda^{(\tau)} - \sum_{t = 1}^{G} \widetilde{\Sigma}_{t}^{(\tau)} \big] - \sum_{t = 1}^{G} (\widetilde{x}_t^{(\tau)}- A^{(\tau)}S^{(\tau)})(\widetilde{x}_t^{(\tau)}- A^{(\tau)}S^{(\tau)})^H \\
\iff G \Lambda^{(\tau)} &= \sum_{t = 1}^{G} \widetilde{\Sigma}_{t}^{(\tau)} + \sum_{t = 1}^{G}  (\widetilde{x}_t^{(\tau)}- A^{(\tau)}S^{(\tau)}) (\widetilde{x}_t^{(\tau)}- A^{(\tau)}S^{(\tau)})^H \\
\iff \Lambda^{(\tau)} &= \frac{1}{G} \sum_{t = 1}^{G} \Big[ \widetilde{\Sigma}_{t}^{(\tau)} +  (\widetilde{x}_t^{(\tau)}- A^{(\tau)}S^{(\tau)}) (\widetilde{x}_t^{(\tau)}- A^{(\tau)}S^{(\tau)})^H \Big]
\end{align*}
Учтем тот факт, что матрица шума оценивается как диагональная:
\begin{align*}
\iff \mathbf{\Lambda}^{(\tau)}  &= \text{diag}\bigg(\frac{1}{G} \sum_{t = 1}^{G} \Big[ \widetilde{\Sigma}_{t}^{(\tau)} +  (\widetilde{x}_t^{(\tau)}- A^{(\tau)}S^{(\tau)}) (\widetilde{x}_t^{(\tau)}- A^{(\tau)}S^{(\tau)})^H \Big]\bigg)
\end{align*}
Шаги повторяются либо до достижения максимального числа итераций либо до сходимости оценок параметров, полученных на соседних двух итерациях.
\clearpage
\begin{center}
\fontsize{16}{20}\selectfont \color{teal}{\textbf{Список источников}}
\end{center}
\end{document}
