\documentclass[11pt]{article}
\usepackage[english,russian]{babel}
\usepackage[utf8]{inputenc}
\usepackage[a4paper, left=2.5cm, right=1.5cm, top=2.5cm, bottom=2.5cm]{geometry}
\usepackage{animate} 
\usepackage{graphicx}
\usepackage{amsmath}
\usepackage{longtable}
\usepackage{amssymb}
\usepackage{physics}
\usepackage{tikz}
\usepackage{comment}
\usepackage{animate} 
\usepackage{graphicx}
\usepackage{amsmath}
\usepackage{longtable}
\usepackage{amssymb}
\usepackage{physics}
\usepackage{tikz}
\usepackage{comment}
\usepackage{colortbl}
%\usepackage{xcolor}
\usepackage[normalem]{ulem}
\usepackage{float}
\usepackage{wrapfig}
\usepackage{cancel}
\usepackage{mathtools}
\usepackage[most]{tcolorbox}
\usepackage[mathscr]{euscript}

\usepackage[dvipsnames]{xcolor}

\DeclarePairedDelimiter\ceil{\lceil}{\rceil}
 
\newcommand{\Expect}{\mathbb{E}}
\newcommand{\Var}{\mathcal{D}}
\newcommand{\Cov}{\mathsf{cov}}
\newcommand{\Norm}{\mathcal{N}}
\newcommand{\NormComplex}{\mathcal{CN}}
\newcommand{\Natural}{\mathbb{N}}
\newcommand{\Real}{\mathbb{R}}
\newcommand{\Complex}{\mathbb{C}}
\newcommand{\Int}{\mathbb{Z}}
\newcommand{\DK}{\mathbf{D}_{KL}}
\DeclarePairedDelimiterX{\infdivx}[2]{(}{)}{%
  #1\;\delimsize\|\;#2%
}
\DeclareMathOperator*{\argmax}{arg\,max}
\DeclareMathOperator*{\argmin}{arg\,min}
\DeclareMathOperator{\Det}{Det}
\newcommand{\infdiv}{D_{KL}\infdivx}
\newcommand\Fontvi{\fontsize{8.2}{7.2}\selectfont}
\newcommand\Fontvia{\fontsize{9}{8}\selectfont}
\newcommand\Fontvib{\fontsize{10.8}{9.6}\selectfont}
\newcommand\Fontvic{\fontsize{8.0}{7.0}\selectfont}
\newcommand{\myitem}{\item[\checkmark]}
%\newcommand{\myitem}{\item[\squares]}

\begin{document}
\begin{center}
\fontsize{20}{23}\selectfont \color{red}{\textbf{ЕМ-алгоритм, стохастическая модель сигнала}}
\end{center}
\begin{center}
\fontsize{14}{18}\selectfont \color{red}{\textbf{M-шаг}}
\end{center}
Требуется найти наилучшую оценку параметров, решив следующую задачу оптимизации:
\begin{equation*}
\begin{gathered}
\Upsilon^{(\tau)}=\argmax_{\Upsilon} \Expect_{(X_m,S)|\mathcal{I}}[\log P(X, S)] = \\
\argmin_{\Upsilon}  \text{T} \Big[\log|\mathbf{\Lambda}| +\Tr \big( \mathbf{\Lambda}^{-1} \Expect_{(X_m,S)|\mathcal{I}}[XX^*] \big) - 2 \Tr\big(\mathbf{\Lambda}^{-1} \text{A}(\theta) \Expect_{(X_m,S)|\mathcal{I}}[X S^*]\big) \\
+ \Tr(\mathbf{\Lambda}^{-1} \text{A}(\theta) \Expect_{(X_m,S)|\mathcal{I}}[SS^*] \text{A}^*(\theta)) + \log |\mathbf{\Gamma}| + \Tr(\mathbf{\Gamma}^{-1} \Expect_{(X_m,S)|\mathcal{I}}[SS^*]) \Big].
\end{gathered}
\end{equation*}
Оценим угловые координаты источников $\theta$:
\begin{equation*}
\begin{gathered}
\theta^{(\tau)}= \argmin_{\theta}\mathcal{Q}_1(\theta|\theta^{(\tau-1)})  = \argmin_{\theta} \Big[ - 2 \Tr\big(\mathbf{\Lambda}^{-1} \text{A}(\theta) \Expect_{(X_m,S)|\mathcal{I}}[X S^*]\big) \\
+ \Tr(\mathbf{\Lambda}^{-1} \text{A}(\theta) \Expect_{(X_m,S)|\mathcal{I}}[S S^*] \text{A}^*(\theta)) \Big] = \\
\argmin_{\theta} ||\mathbf{\Lambda}^{-1/2} (\Expect_{(X_m,S)|\mathcal{I}}[X S^*]-\text{A}(\theta)\Expect_{(X_m,S)|\mathcal{I}}[S S^*]) ||_F^2.
\end{gathered}
\end{equation*}
На практике, если $\text{A}$ соответствует антенной решетки типа ULA, удобно искать оптимальный $u=\sin(\theta)$, а затем находить $\theta$ как $\arcsin(u)$. Для удобства введем новые обозначения: $C=\Expect_{(X_m,S)|\mathcal{I}}[X S^*], D=\Expect_{(X_m,S)|\mathcal{I}}[S S^*]$. Соответственно минимизации подлежит функция
\begin{equation}
\mathcal{Q}_1(u|u^{\tau-1)}) = ||\mathbf{\Lambda}^{-1/2} (C-\widetilde{A}(u)D) ||_F^2,
\end{equation}
где
\begin{equation}
\widetilde{A}(u) = \begin{bmatrix}
1&1&\dots&1\\
e^{-2j\pi \frac{\text{d}}{\lambda}u_1}& e^{-2j\pi \frac{\text{d}}{\lambda}u_2}&\dots&e^{-2j\pi \frac{\text{d}}{\lambda}u_\text{K}}\\
\dots&\dots&\ddots&\vdots\\
e^{-2j\pi (\text{L}-1) \frac{\text{d}}{\lambda}u_1}& e^{-2j\pi (\text{L}-1) \frac{\text{d}}{\lambda}u_2}&\dots&e^{-2j\pi (\text{L}-1) \frac{\text{d}}{\lambda}u_\text{K}}\\
\end{bmatrix}.
\nonumber
\end{equation}
Для ускорения оптимизации можно оптимизировать не $\mathcal{Q}_1(u|u^{(\tau-1)})$, а суррогатную функцию $\mathcal{G}(u|u^{(\tau-1)})$, построенную так:
\begin{equation}
\mathcal{G}(u|u^{(\tau-1)}) = \mathcal{Q}_1(u^{(\tau-1)}|u^{(\tau-1)}) + \grad \mathcal{Q}_1(u^{(\tau-1)}|u^{(\tau-1)})^T (u-u^{(\tau-1)}) + \frac{1}{2} (u-u^{(\tau-1)})^T \mathbf{H} (u-u^{(\tau-1)}),
\end{equation}
где
\begin{equation}
\mathbf{H} = \begin{bmatrix}
\max(|\grad_1 \mathcal{Q}_1(u^{(\tau-1)}|u^{(\tau-1)})|,\varepsilon)&0&\dots&0\\
0& \max(|\grad_2 \mathcal{Q}_1(u^{(\tau-1)}|u^{(\tau-1)})|,\varepsilon)&\dots&0\\
\dots&\dots&\ddots&\vdots\\
0& 0&\dots& \max(|\grad_\text{K} \mathcal{Q}_1(u^{(\tau-1)}|u^{(\tau-1)})|,\varepsilon)\\
\end{bmatrix}.
\nonumber
\end{equation}
Теперь эту суррогатную функцию надо минимизировать (или, эквивалентно, найти направление шага $d^{(\tau)}=u-u^{(\tau-1)}$):
\begin{equation}
\min_u \mathcal{G}(u|u^{(\tau-1)}).
\end{equation}
Подставим $u = u^{(\tau-1)}+d$:
\begin{equation}
\mathcal{G}(u^{(\tau-1)} + d| u^{(\tau-1)}) = \mathcal{Q}_1(u^{(\tau-1)}| u^{(\tau-1)}) + \grad \mathcal{Q}_1 (u^{(\tau-1)}| u^{(\tau-1)})^Td + \frac{1}{2}d^T\mathbf{H}d.
\end{equation}
\clearpage
Для нахождения минимума берем градиент по $d$:
\begin{equation}
\grad_d \mathcal{G}(u^{(\tau-1)} + d| u^{(\tau-1)}) =  \grad \mathcal{Q}_1(u^{(\tau-1)}| u^{(\tau-1)})   + \mathbf{H}d,
\end{equation}
при $\grad_d \mathcal{G}=0$ будет выполняться:
\begin{equation}
0 =  \grad \mathcal{Q}_1(u^{(\tau-1)}| u^{(\tau-1)})   + \mathbf{H}d,
\end{equation}
решаем относительно $d^{(\tau)}$:
\begin{equation}
d^{(\tau)} = -\mathbf{H}^{-1}  \grad \mathcal{Q}_1(u^{(\tau-1)}| u^{(\tau-1)}).
\end{equation}
Далее используется backtracking line search для гарантии невозрастания $\mathcal{Q}_1$. \\
Оценим ковариацию сигналов $\mathbf{\Gamma}$:
\begin{equation*}
\begin{gathered}
\mathbf{\Gamma}^{(\tau)}= \argmin_{\mathbf{\Gamma}} \mathcal{Q}_2(\mathbf{\Gamma} | \mathbf{\Gamma}^{(\tau-1)})
 = \text{T}\bigg[ \log |\mathbf{\Gamma}| + \Tr(\mathbf{\Gamma}^{-1}\Expect_{(X_m,S)|\mathcal{I}}[S S^*])\bigg].
\end{gathered}
\end{equation*}
Определим точку, где производная данной функции принимает значение 0, и, таким образом, находим минимум функции:
\begin{equation*}
\begin{gathered}
\frac{\partial}{\partial \mathbf{\Gamma}}\log (\Det (\mathbf{\Gamma})) = \mathbf{\Gamma}^{-1}, \\
\frac{\partial}{\partial \mathbf{\Gamma}}\Tr(\mathbf{\Gamma}^{-1}\Expect_{(X_m,S)|\mathcal{I}}[S S^*])= -\mathbf{\Gamma}^{-1}\Expect_{(X_m,S)|\mathcal{I}}[S S^*]\mathbf{\Gamma}^{-1}, \\
\frac{\partial \mathcal{Q}_2(\mathbf{\Gamma})}{\partial \mathbf{\Gamma}} = \mathbf{\Gamma}^{-1}-\mathbf{\Gamma}^{-1}\Expect_{(X_m,S)|\mathcal{I}}[S S^*]\mathbf{\Gamma}^{-1}.
\end{gathered}
\end{equation*}
Приравняем производную к нулю (функция по $\mathbf{\Gamma}$ выпукла):
\begin{equation*}
\mathbf{O} = \mathbf{\Gamma}^{-1}-\mathbf{\Gamma}^{-1}\Expect_{(X_m,S)|\mathcal{I}}[S S^*]\mathbf{\Gamma}^{-1} \Rightarrow \Expect_{(X_m,S)|\mathcal{I}}[S S^*] = \mathbf{\Gamma} \Rightarrow \mathbf{\Gamma}^{(\tau)} = \Expect_{(X_m,S)|\mathcal{I}}[S S^*].
\end{equation*}
Предполагая, что сигналы некоррелированны, будем использовать лишь диагональное приближение матрицы, приравняв элементы вне главной диагонали к нулю:
\begin{equation}
\mathbf{\Gamma}^{(\tau)} =  \mathcal{D} \Big[\Expect_{(X_m,S)|\mathcal{I}}[S S^*] \Big].
\end{equation}
Обновляем оценку ковариации наблюдений с учетом полученных оценок параметров:
\begin{equation}
\hat{\text{R}}^{(\tau)} = \text{A}(\theta^{(\tau)})\mathbf{\Gamma}\text{A}^*(\theta^{(\tau)}) + \mathbf{\Lambda}.
\end{equation}
\end{document}

