\documentclass[11pt]{article}
\usepackage[english,russian]{babel}
\usepackage[utf8]{inputenc}
\usepackage[a4paper, left=2.5cm, right=1.5cm, top=2.5cm, bottom=2.5cm]{geometry}
\usepackage{animate} 
\usepackage{graphicx}
\usepackage{amsmath}
\usepackage{longtable}
\usepackage{amssymb}
\usepackage{physics}
\usepackage{tikz}
\usepackage{comment}
\usepackage{animate} 
\usepackage{graphicx}
\usepackage{amsmath}
\usepackage{longtable}
\usepackage{amssymb}
\usepackage{physics}
\usepackage{tikz}
\usepackage{comment}
\usepackage{colortbl}
%\usepackage{xcolor}
\usepackage[normalem]{ulem}
\usepackage{float}
\usepackage{wrapfig}
\usepackage{cancel}
\usepackage{mathtools}
\usepackage[most]{tcolorbox}
\usepackage[mathscr]{euscript}

\usepackage[dvipsnames]{xcolor}

\DeclarePairedDelimiter\ceil{\lceil}{\rceil}
 
\newcommand{\Expect}{\mathbb{E}}
\newcommand{\Var}{\mathcal{D}}
\newcommand{\Cov}{\mathsf{cov}}
\newcommand{\Norm}{\mathcal{N}}
\newcommand{\NormComplex}{\mathcal{CN}}
\newcommand{\Natural}{\mathbb{N}}
\newcommand{\Real}{\mathbb{R}}
\newcommand{\Complex}{\mathbb{C}}
\newcommand{\Int}{\mathbb{Z}}
\newcommand{\DK}{\mathbf{D}_{KL}}
\DeclarePairedDelimiterX{\infdivx}[2]{(}{)}{%
  #1\;\delimsize\|\;#2%
}
\DeclareMathOperator*{\argmax}{arg\,max}
\DeclareMathOperator*{\argmin}{arg\,min}
\DeclareMathOperator{\Det}{Det}
\newcommand{\infdiv}{D_{KL}\infdivx}
\newcommand\Fontvi{\fontsize{8.2}{7.2}\selectfont}
\newcommand\Fontvia{\fontsize{9}{8}\selectfont}
\newcommand\Fontvib{\fontsize{10.8}{9.6}\selectfont}
\newcommand\Fontvic{\fontsize{8.0}{7.0}\selectfont}
\newcommand{\myitem}{\item[\checkmark]}
%\newcommand{\myitem}{\item[\squares]}

\begin{document}
\begin{center}
\fontsize{20}{23}\selectfont \color{red}{\textbf{Об апостериорном распределении сигналов}}
\end{center}
\begin{equation}
\text{Cov}(U,V) = \Expect[\text{Cov}(U,V|Y)] + \text{Cov}(\Expect[U|Y], \Expect[V|Y]).
\end{equation}
Если $U=V$, то:
\begin{equation}
\text{Cov}(X) = \Expect[\text{Cov}(X|Y)] + \text{Cov}(\Expect[X|Y]).
\end{equation}
Применим это тождество для случая условной вероятности относительно $\sigma$-алгебры $\sigma(Z)$, причем $\sigma(Z) \subseteq \sigma(Y)$, предполагается, что $Z=CY$, где $C$ -- булев селектор:
\begin{equation*}
\begin{gathered}
\text{Cov}(X|Z) = \Expect[\text{Cov}(X|Y,Z)|Z] + \text{Cov}(\Expect[X|Y,Z]|Z).
\end{gathered}
\end{equation*}
Вектора $X$, $Y$, $Z$ предполагаются комплексными гауссовскими (случай круговой симметрии), зависимость между $Z$ и $Y$ линейная. Для линейной гауссовской модели знание $Z$ не уменьшает условную ковариацию $X|Y$ и не влияет на условное математическое ожидание $\Expect[X|Y,Z]$, поскольку $Z$ не содержит никакой новой информации, которой не было бы в $Y$:
\begin{equation*}
\begin{gathered}
\text{Cov}(X|Y,Z)=\text{Cov}(X|Y), \\
\Expect[X|Y,Z] = \Expect[X|Y].
\end{gathered}
\end{equation*}
Получаем:
\begin{equation}
\text{Cov}(X|Z) = \Expect[\text{Cov}(X|Y)|Z] + \text{Cov}(\Expect[X|Y]|Z).
\end{equation}
\end{document}

