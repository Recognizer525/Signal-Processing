\documentclass[11pt]{article}
\usepackage[english,russian]{babel}
\usepackage[utf8]{inputenc}
\usepackage[a4paper, left=2.5cm, right=1.5cm, top=2.5cm, bottom=2.5cm]{geometry}
\usepackage{animate} 
\usepackage{graphicx}
\usepackage{amsmath}
\usepackage{longtable}
\usepackage{amssymb}
\usepackage{physics}
\usepackage{tikz}
\usepackage{comment}
\usepackage{animate} 
\usepackage{graphicx}
\usepackage{amsmath}
\usepackage{longtable}
\usepackage{amssymb}
\usepackage{physics}
\usepackage{tikz}
\usepackage{comment}
\usepackage{colortbl}
%\usepackage{xcolor}
\usepackage[normalem]{ulem}
\usepackage{float}
\usepackage{wrapfig}
\usepackage{cancel}
\usepackage{mathtools}
\usepackage[most]{tcolorbox}
\usepackage[mathscr]{euscript}

\usepackage[dvipsnames]{xcolor}

\DeclarePairedDelimiter\ceil{\lceil}{\rceil}
 
\newcommand{\Expect}{\mathbb{E}}
\newcommand{\Var}{\mathcal{D}}
\newcommand{\Cov}{\mathsf{cov}}
\newcommand{\Norm}{\mathcal{N}}
\newcommand{\NormComplex}{\mathcal{CN}}
\newcommand{\Natural}{\mathbb{N}}
\newcommand{\Real}{\mathbb{R}}
\newcommand{\Complex}{\mathbb{C}}
\newcommand{\Int}{\mathbb{Z}}
\newcommand{\DK}{\mathbf{D}_{KL}}
\DeclarePairedDelimiterX{\infdivx}[2]{(}{)}{%
  #1\;\delimsize\|\;#2%
}
\DeclareMathOperator*{\argmax}{arg\,max}
\DeclareMathOperator*{\argmin}{arg\,min}
\DeclareMathOperator{\Det}{Det}
\newcommand{\infdiv}{D_{KL}\infdivx}
\newcommand\Fontvi{\fontsize{8.2}{7.2}\selectfont}
\newcommand\Fontvia{\fontsize{9}{8}\selectfont}
\newcommand\Fontvib{\fontsize{10.8}{9.6}\selectfont}
\newcommand\Fontvic{\fontsize{8.0}{7.0}\selectfont}
\newcommand{\myitem}{\item[\checkmark]}
%\newcommand{\myitem}{\item[\squares]}

\begin{document}
\begin{center}
\fontsize{20}{23}\selectfont \color{red}{\textbf{ЕМ-алгоритм, стохастическая модель сигнала}}
\end{center}
\begin{center}
\fontsize{16}{20}\selectfont \color{teal}{\textbf{Постановка проблемы}}
\end{center}
Предположим, имеется линейная антенная решетка, состоящая из $\text{L}$ сенсоров, которая принимает сигналы, направленные из $\text{K}$  источников, причем $\text{K} < \text{L}$. Этим источникам соответствуют угловые координаты (DoA) $\theta$, практически не изменяющиеся во времени. По итогам измерений было получено $\text{T}$ снимков полученного сигнала, причем ввиду стохастических технических сбоев, связанных с сенсорами, большая часть таких снимков содержит помимо надежных данных ненадежные, которые в рамках данной задачи рассматриваются как пропуски. Пусть $X$ --- набор наблюдений, полученных сенсорами в моменты времени $t=1,\ldots,\text{T}$, $X_t$ соответствует наблюдению в момент времени $t$, через $x$ и $x_t$ будем обозначать реализации полного набора наблюдений и наблюдения в отдельный момент времени $t$  соответственно. Ввиду наличия пропусков в данных, будем считать, что набор наблюдений $X$ состоит из доступной части $X_o = \{X_{t, o_t}\}_{t=1}^\text{T}$ и недоступной: $X_m = \{X_{t, m_t}\}_{t=1}^\text{T}$, причем $o_t \cup m_t = \{1,\ldots,\text{T}\}, o_t \cap m_t = \varnothing,  \forall t \in \{1,\ldots,\text{T}\}$. Предполагается, что $\nexists o_t: o_t = \varnothing$, т.е. нет таких наблюдений, которые состоят лишь из недоступной части. \\ Набор $X$ является результатом следующей модели наблюдений:
\begin{equation}
X = \text{A} S + N,
\end{equation}
где
\begin{equation}
X=[X_1,\ldots, X_\text{T}], S=[S_1,\ldots, S_\text{T}], N=[N_1,\ldots, N_\text{T}],
\end{equation}
и $X_t \in \Complex^{\text{L} \times \text{T}}, S_t \in \Complex^{\text{K} \times \text{T}}, N_t \in \Complex^{\text{L} \times \text{T}}$ -- векторы-столбцы, соответствующие наблюдениям, источникам и шумам в момент времени  $t = 1, \ldots, \text{T}$, $\text{A}$ -- матрица векторов направленности для равномерного линейного массива:
\begin{equation}
\text{A}(\theta) = \begin{bmatrix}
1&1&\dots&1\\
e^{-2j\pi \frac{\text{d}}{\lambda}\sin(\theta_1)}& e^{-2j\pi \frac{\text{d}}{\lambda}\sin(\theta_2)}&\dots&e^{-2j\pi \frac{\text{d}}{\lambda}\sin(\theta_\text{K})}\\
\dots&\dots&\ddots&\vdots\\
e^{-2j\pi (\text{L}-1) \frac{\text{d}}{\lambda}\sin(\theta_1)}& e^{-2j\pi (\text{L}-1) \frac{\text{d}}{\lambda}\sin(\theta_2)}&\dots&e^{-2j\pi (\text{L}-1) \frac{\text{d}}{\lambda}\sin(\theta_\text{K})}\\
\end{bmatrix}.
\nonumber
\end{equation}
Сигналы, испускаемые источниками, шумы на сенсорах и наблюдения предполагаются стохастическими: $S_t \sim \NormComplex(\mathbf{O}_{\text{K} \times 1},\mathbf{\Gamma})$, $N_t \sim \NormComplex(\mathbf{O}_{\text{L} \times 1}, \mathbf{\Lambda})$, $ X_t \sim \NormComplex(\mathbf{O}_{\text{L} \times 1}, \text{A}\mathbf{\Gamma}\text{A}^* + \mathbf{\Lambda})$, $t=1,\ldots,\text{T}$. Матрицы $\mathbf{\Gamma}$ и $\mathbf{\Lambda}$ предполагаются диагональными, т.е. сигналы не коррелированы между собой, шумы также не коррелированы между собой, корреляция между сигналами и шумами также отсутствует. Сигналы предполагаются узкополосными.
Составим EM-алгоритм для случая известного шума.
\clearpage
\begin{center}
\fontsize{16}{20}\selectfont \color{teal}{\textbf{ЕМ-алгоритм для известного шума}}
\end{center}
Воспользуемся EM-алгоритмом для того, чтобы определить значения параметров $\Upsilon = (\theta, \mathbf{\Gamma})$, пропущенные значения $X_m=\{X_{t, m_t}\}_{t=1}^\text{T}$ и сигналы $S$ рассматриваются как латентные переменные. Наблюдения $X_t$, $t=1,\ldots,\text{T}$ предполагаются независимыми и одинаково распределенными.
\begin{center}
\fontsize{14}{18}\selectfont \color{red}{\textbf{Инициализация параметров}}
\end{center}
Оценим вектор угловых координат источников $\theta^{(0)}$ следующим образом:
\begin{enumerate}
\item
Выберем число $\nu$, которое будет соответствовать первому компоненту вектора $\theta^{(0)}$:
\begin{equation}
\nu \sim \mathcal{U}([-\pi;\pi]);
\end{equation}
\item
Оценим компоненты вектора $\theta^{(0)}$ так:  $\theta^{(0)}_i = (\nu + (i-1)\cdot \frac{2\pi}{\text{K}})\, \text{mod} \, 2\pi, i = 1,\ldots,\text{K}$. При этом,  $a \, \text{mod} \, b = a - b \cdot \lfloor \frac{a}{b} \rfloor$.
\end{enumerate}
Начальную оценку ковариационной матрицы наблюдений $\hat{\text{R}}^{(0)}$ получаем на основе доступных данных. Начальную ковариацию сигналов $\Gamma^{(0)}$ получаем на основе метода наименьших квадратов с помощью $\theta^{(0)}, \hat{\text{R}}^{(0)}$.
\begin{center}
\fontsize{14}{18}\selectfont \color{red}{\textbf{Е-шаг}}
\end{center}
Требуется найти математическое ожидание полного правдоподобия с учетом  текущей оценки параметров и апостериорного совместного распределения пропущенных значений в наблюдениях $X_m$ и сигналов $S$ 
\begin{equation}
 \Expect_{(X_m,S)|X_o=x_o, \Upsilon^{(\tau-1)}}[\log P(X, S)].
\end{equation}
Преобразуем выражение, учитывая тот факт, что наблюдения являются независимыми, и обозначив через $\mathcal{I}$ условную информацию $X_o=x_o, \Upsilon^{(\tau-1)}$ и через $\mathcal{I}_t$ условную информацию $X_{t,o_t}=x_{t,o_t}, \Upsilon^{(\tau-1)}$:
\begin{equation*}
\begin{gathered}
 \Expect_{(X_m,S)|\mathcal{I}}[\log P(X, S)] = \\
 \Expect_{(X_m,S)|\mathcal{I}}[\log P(X|S) + \log P(S)] = \\
\Big[\sum_{t=1}^T  \Expect_{(X_{t,m_t},S_t)|\mathcal{I}_t} \log [P(X_t|S_t)] + \sum_{t=1}^T \Expect_{(X_{t,m_t},S_t)|\mathcal{I}_t}[\log P(S_t)]\Big] = \\
- \text{T} \Big[\log|\mathbf{\Lambda}| +\Tr \big( \mathbf{\Lambda}^{-1} \Expect_{(X_m,S)|\mathcal{I}}[XX^*] \big) - 2 \Tr\big(\mathbf{\Lambda}^{-1} \text{A}(\theta) \Expect_{(X_m,S)|\mathcal{I}}[X S^*]\big) \\
+ \Tr(\mathbf{\Lambda}^{-1} \text{A}(\theta) \Expect_{(X_m,S)|\mathcal{I}}[SS^*] \text{A}^*(\theta)) + \log |\mathbf{\Gamma}| + \Tr(\mathbf{\Gamma}^{-1} \Expect_{(X_m,S)|\mathcal{I}}[SS^*]) \Big].
\end{gathered}
\end{equation*}
Выражение $\Expect_{(X_{t,m_t},S_t)|\mathcal{I}_t}[\cdot]$  соответствуют случаю максимально полного набора латентных переменных: некоторые наблюдения в наборе могут не содержать пропуски, и тогда это выражение упрощается до $\Expect_{S_t|\mathcal{I}_t}[\cdot]$.
Для нахождения условных моментов, указанных в формуле выше, требуется найти апостериорное распределение скрытых переменных. 
\begin{center}
\fontsize{12}{16}\selectfont \color{Purple}{\textbf{Случай неполных наблюдений}}
\end{center}
Воспользуемся формулой произведения плотностей:
\begin{gather}
P(X_{t,m_t},S_t|\mathcal{I}_t) = P(X_{t,m_t}|\mathcal{I}_t) \cdot P(S_t|X_{t,m_t}, \mathcal{I}_t). 
\end{gather}
Сначала найдем апостериорное распределение $P(X_m|\mathcal{I})$. Для достижения этой цели, для каждой пары $ \{ (o_t, m_t): m_t \ne \varnothing \}$ создадим разбиение оценки ковариационной матрицы наблюдений $\hat{\text{R}}$ на блоки, индуцированное этим разбиением множества индексов, оно имеет следующий вид:
\begin{equation}
\hat{\text{R}} =
\begin{pmatrix}
\hat{\text{R}}_{o_t, o_t} & \hat{\text{R}}_{o_t, m_t}\\
\hat{\text{R}}_{m_t, o_t} & \hat{\text{R}}_{m_t, m_t}
\end{pmatrix},
\end{equation}
где каждый блок определяется как
\begin{equation*}
\hat{\text{R}}_{a,b} = (\hat{\text{R}}_{ij})_{i \in a, j \in b}.
\end{equation*}
Для каждого наблюдения, содержащего пропуски, требуется найти апостериорное распределение пропущенных значений, $P(X_{t, m_t}|X_{t, o_t}=x_{t, o_t},\Upsilon^{(\tau-1)}), t=1,\ldots,\text{T}, m_t \ne \varnothing$ . Обозначим через $\mathcal{I}_t$ условную информацию $X_{t, o_t}=x_{t, o_t},\Upsilon^{(\tau-1)}$.
Параметры апостериорного распределения $P(X_{t, m_t}|\mathcal{I}_t), t=1,\ldots,\text{T}$ на итерации $\tau$ можно найти следующим образом:
\begin{equation}
\left\{ \begin{aligned} 
\Expect[X_{t, m_t}|\mathcal{I}_t] &= \hat{\text{R}}_{m_t, o_t}\left(\hat{\text{R}}_{o_t, o_t}\right)^{-1}\cdot x_{t, o_t}, \\
\Sigma_{X_{t, m_t}|\mathcal{I}_t} &= \hat{\text{R}}_{m_t, m_t}-\hat{\text{R}}_{m_t, o_t}\left(\hat{\text{R}}_{o_t, o_t}\right)^{-1}\hat{\text{R}}_{o_t, m_t},
\end{aligned} \right.
\end{equation}
где $\hat{\text{R}}_{o_t, o_t}= \hat{\text{R}}_{o_t, o_t}^{(\tau-1)}$, 
$\hat{\text{R}}_{o_t, m_t}= \hat{\text{R}}_{o_t, m_t}^{(\tau-1)}$, 
$\hat{\text{R}}_{m_t, o_t}= \hat{\text{R}}_{m_t, o_t}^{(\tau-1)}$,
$\hat{\text{R}}_{m_t, m_t}= \hat{\text{R}}_{m_t, m_t}^{(\tau-1)}$.\\
Для каждого наблюдения $X_t$, содержащего пропуски, оценим условный второй начальный момент $\Expect[X_t X_t^*|\mathcal{I}_t]$:
\begin{equation*}
\begin{gathered}
\Expect[X_t X_t^*|\mathcal{I}_t] = \Expect [X_t | \mathcal{I}_t] \cdot \Expect [X_t | \mathcal{I}_t]^* +
\begin{pmatrix}
\mathbf{O}_{o_t, o_t} & \mathbf{O}_{o_t, m_t} \\
\mathbf{O}_{m_t, o_t} & \Sigma_{X_{t, m_t}|\mathcal{I}_t} \\
\end{pmatrix},
\end{gathered}
\end{equation*}
где $\mathbf{O}_{o_t, o_t}$, $\mathbf{O}_{o_t, m_t}$, $\mathbf{O}_{m_t, o_t}$ -- нулевые блочные матрицы. Разбиение указанной матрицы на четыре блока, три из которых состоят из нулей, индукцировано разбиением множества индексов $\{1,\ldots,\text{L}\}$ на множества $o_t, m_t$. \\
Параметры апостериорного распределения $P(S_t|\mathcal{I}_t, X_{t, m_t}), t = 1,\ldots, \text{T}$ можно найти исходя из следующих формул, если $m_t \ne \varnothing$:
\begin{equation}
\left\{ \begin{aligned} 
\Expect[S_t|\mathcal{I}_t, X_{t,m_t}] &= \mathbf{\Gamma}^*\text{A}^* \hat{\text{R}}^{-1}\Expect [X_t | \mathcal{I}_t], \\
\Sigma_{S_t|\mathcal{I}_t, X_{t, m_t}} &= \mathbf{\Gamma} - \mathbf{\Gamma}^*\text{A}^*  \hat{\text{R}}^{-1}\text{A}\mathbf{\Gamma} + \mathbf{\Gamma}\text{A}^* \hat{\text{R}}^{-1}\Expect[X_t X_t^*|\mathcal{I}_t]\hat{\text{R}}^{-1} \text{A}\mathbf{\Gamma},
\end{aligned} \right.
\end{equation}
где $\text{A}=\text{A}(\theta^{(\tau-1)})$, $\mathbf{\Gamma}=\mathbf{\Gamma}^{(\tau-1)}$, $\hat{\text{R}}=\hat{\text{R}}^{(\tau-1)}$. \\
Оценим $\Expect_{(X_{t,m_t},S_t)|\mathcal{I}_t}[S_t S_t^*]$:
\begin{equation}
\Expect_{(X_{t,m_t},S_t)|\mathcal{I}_t}[S_t S_t^*] = \Expect[S_t|\mathcal{I}_t, X_{t,m_t}] \cdot \Expect[S_t|\mathcal{I}_t, X_{t,m_t}]^* + \Sigma_{S_t|\mathcal{I}_t, X_{t, m_t}},
\end{equation}
Оценим $\Expect_{(X_{t,m_t},S_t)|\mathcal{I}_t}[X_t S_t^*]$:
\begin{equation}
\begin{gathered}
\Expect_{(X_{t,m_t},S_t)|\mathcal{I}_t}[X_t S_t^*] = \Expect[X_t X_t^*|\mathcal{I}_t]\hat{\text{R}}^{-1}\text{A}\mathbf{\Gamma} + \Expect[X_t|\mathcal{I}_t] \cdot \Expect[S_t|\mathcal{I}_t, X_{t,m_t}]^*,
\end{gathered}
\end{equation}
где $\text{A}=\text{A}(\theta^{(\tau-1)})$, $\hat{\text{R}}=\hat{\text{R}}^{(\tau-1)}$, $\mathbf{\Gamma}=\mathbf{\Gamma}^{(\tau-1)}$. \\
\begin{center}
\fontsize{12}{16}\selectfont \color{Purple}{\textbf{Случай полных наблюдений}}
\end{center}
Если $m_t = \varnothing$, то апостериорное распределение упрощается следующим образом: $P(X_{t,m_t},S_t|\mathcal{I}_t)=P(S_t|\mathcal{I}_t)$.
Параметры апостериорного распределения $P(S_t|\mathcal{I}_t), t = 1,\ldots, \text{T}$:
\begin{equation}
\left\{ \begin{aligned} 
\Expect[S_t|\mathcal{I}_t] &= \mathbf{\Gamma}^*\text{A}^* \hat{\text{R}}^{-1}\Expect [X_t | \mathcal{I}_t], \\
\Sigma_{S_t|\mathcal{I}_t} &= \mathbf{\Gamma} - \mathbf{\Gamma}^*\text{A}^*  \hat{\text{R}}^{-1}\text{A}\mathbf{\Gamma}.
\end{aligned} \right.
\end{equation}
При этом:
\begin{equation}
\Expect_{S_t|\mathcal{I}_t}[X_t X_t^*] = \Expect[X_t|\mathcal{I}_t] \cdot \Expect[X_t|\mathcal{I}_t]^* = x_t x_t^*,
\end{equation}
\begin{equation}
\Expect_{S_t|\mathcal{I}_t}[X_t S_t^*] = \Expect[X_t|\mathcal{I}_t] \cdot \Expect[S_t|\mathcal{I}_t]^* = x_t \cdot \Expect[S_t|\mathcal{I}_t]^* ,
\end{equation}
\begin{equation}
\Expect_{S_t|\mathcal{I}_t}[S_t S_t^*] = \Expect[S_t|\mathcal{I}_t] \cdot \Expect[S_t|\mathcal{I}_t]^* + \Sigma_{S_t|\mathcal{I}_t}.
\end{equation}
\begin{center}
\fontsize{12}{16}\selectfont \color{Purple}{\textbf{Агрегация}}
\end{center}
Теперь оценим вторые начальные моменты  $\Expect_{(X_m,S)|\mathcal{I}}[X X^*]$,  $\Expect_{(X_m,S)|\mathcal{I}}[S S^*]$, $\Expect_{(X_m,S)|\mathcal{I}}[X S^*]$:
\begin{equation}
\Expect_{(X_m,S)|\mathcal{I}}[XX^*] = \frac{1}{\text{T}}\sum_{t=1}^T \Expect[X_t X_t^*|\mathcal{I}_t],
\end{equation}
\begin{equation}
\Expect_{(X_m,S)|\mathcal{I}}[X S^*] =  \frac{1}{\text{T}} \Big[ \sum_{t \in T_1} \Expect_{(X_{t,m_t},S_t)|\mathcal{I}_t}[S_t S_t^*] + \sum_{t \in T_2} \Expect_{S_t|\mathcal{I}_t}[S_t S_t^*] \Big],
\end{equation}
\begin{equation}
\Expect_{(X_m,S)|\mathcal{I}}[X S^*] =  \frac{1}{\text{T}} \Big[ \sum_{t \in T_1} \Expect_{(X_{t,m_t},S_t)|\mathcal{I}_t}[X_t S_t^*] + \sum_{t \in T_2} \Expect_{S_t|\mathcal{I}_t}[X_t S_t^*] \Big],
\end{equation}
где $T_1 = \{y \in \Natural: m_y \ne \varnothing, 1 \le y \le T\}$, $T_2 = \{y \in \Natural: m_y = \varnothing, 1 \le y \le T\}$.
\begin{center}
\fontsize{14}{18}\selectfont \color{red}{\textbf{M-шаг}}
\end{center}
Требуется найти наилучшую оценку параметров, решив следующую задачу оптимизации:
\begin{equation*}
\begin{gathered}
\Upsilon^{(\tau)}=\argmax_{\Upsilon} \Expect_{(X_m,S)|\mathcal{I}}[\log P(X, S)] = \\
\argmin_{\Upsilon}  \text{T} \Big[\log|\mathbf{\Lambda}| +\Tr \big( \mathbf{\Lambda}^{-1} \Expect_{(X_m,S)|\mathcal{I}}[XX^*] \big) - 2 \Tr\big(\mathbf{\Lambda}^{-1} \text{A}(\theta) \Expect_{(X_m,S)|\mathcal{I}}[X S^*]\big) \\
+ \Tr(\mathbf{\Lambda}^{-1} \text{A}(\theta) \Expect_{(X_m,S)|\mathcal{I}}[SS^*] \text{A}^*(\theta)) + \log |\mathbf{\Gamma}| + \Tr(\mathbf{\Gamma}^{-1} \Expect_{(X_m,S)|\mathcal{I}}[SS^*]) \Big].
\end{gathered}
\end{equation*}
Оценим угловые координаты источников $\theta$:
\begin{equation*}
\begin{gathered}
\theta^{(\tau)}= \argmin_{\theta}\mathcal{Q}_1(\theta|\theta^{(\tau-1)})  = \argmin_{\theta} \Big[ - 2 \Tr\big(\mathbf{\Lambda}^{-1} \text{A}(\theta) \Expect_{(X_m,S)|\mathcal{I}}[X S^*]\big) \\
+ \Tr(\mathbf{\Lambda}^{-1} \text{A}(\theta) \Expect_{(X_m,S)|\mathcal{I}}[S S^*] \text{A}^*(\theta)) \Big] = \\
\argmin_{\theta} ||\mathbf{\Lambda}^{-1/2} (\Expect_{(X_m,S)|\mathcal{I}}[X S^*]-\text{A}(\theta)\Expect_{(X_m,S)|\mathcal{I}}[S S^*]) ||_F^2.
\end{gathered}
\end{equation*}
Эту задачу можно решить численно, подробности приведены в приложении 2. \\
Оценим ковариацию сигналов $\mathbf{\Gamma}$:
\begin{equation*}
\begin{gathered}
\mathbf{\Gamma}^{(\tau)}= \argmin_{\mathbf{\Gamma}} \mathcal{Q}_2(\mathbf{\Gamma} | \mathbf{\Gamma}^{(\tau-1)})
 = \text{T}\bigg[ \log |\mathbf{\Gamma}| + \Tr(\mathbf{\Gamma}^{-1}\Expect_{(X_m,S)|\mathcal{I}}[S S^*])\bigg].
\end{gathered}
\end{equation*}
Определим точку, где производная данной функции принимает значение 0, и, таким образом, находим минимум функции:
\begin{equation*}
\begin{gathered}
\frac{\partial}{\partial \mathbf{\Gamma}}\log (\Det (\mathbf{\Gamma})) = \mathbf{\Gamma}^{-1}, \\
\frac{\partial}{\partial \mathbf{\Gamma}}\Tr(\mathbf{\Gamma}^{-1}\Expect_{(X_m,S)|\mathcal{I}}[S S^*])= -\mathbf{\Gamma}^{-1}\Expect_{(X_m,S)|\mathcal{I}}[S S^*]\mathbf{\Gamma}^{-1}, \\
\frac{\partial \mathcal{Q}_2(\mathbf{\Gamma})}{\partial \mathbf{\Gamma}} = \mathbf{\Gamma}^{-1}-\mathbf{\Gamma}^{-1}\Expect_{(X_m,S)|\mathcal{I}}[S S^*]\mathbf{\Gamma}^{-1}.
\end{gathered}
\end{equation*}
Приравняем производную к нулю (функция по $\mathbf{\Gamma}$ выпукла):
\begin{equation*}
\mathbf{O} = \mathbf{\Gamma}^{-1}-\mathbf{\Gamma}^{-1}\Expect_{(X_m,S)|\mathcal{I}}[S S^*]\mathbf{\Gamma}^{-1} \Rightarrow \Expect_{(X_m,S)|\mathcal{I}}[S S^*] = \mathbf{\Gamma} \Rightarrow \mathbf{\Gamma}^{(\tau)} = \Expect_{(X_m,S)|\mathcal{I}}[S S^*].
\end{equation*}
Предполагая, что сигналы некоррелированны, будем использовать лишь диагональное приближение матрицы, приравняв элементы вне главной диагонали к нулю:
\begin{equation}
\mathbf{\Gamma}^{(\tau)} =  \mathcal{D} \Big[\Expect_{(X_m,S)|\mathcal{I}}[S S^*] \Big].
\end{equation}
Обновляем оценку ковариации наблюдений с учетом полученных оценок параметров:
\begin{equation}
\hat{\text{R}}^{(\tau)} = \text{A}(\theta^{(\tau)})\mathbf{\Gamma}\text{A}^*(\theta^{(\tau)}) + \mathbf{\Lambda}.
\end{equation}
\clearpage
\begin{center}
\fontsize{16}{20}\selectfont \color{teal}{\textbf{Результаты численного эксперимента}}
\end{center}
\clearpage
\begin{center}
\fontsize{16}{20}\selectfont \color{teal}{\textbf{Список источников}}
\end{center}
\begin{enumerate}
\item
Dempster A. P., Laird N. M., Rubin D. B. Maximum likelihood from incomplete data via the EM algorithm // Journal of the Royal Statistical Society: Series B (Methodological). – 1977. – Vol. 39. – No. 1.  – P. 1–38.
\item
Wu C. F. J. On the convergence properties of the EM algorithm // The Annals of Statistics. – 1983. – Vol. 11. –  No. 1. –  P. 95–103.
\item
Louis T. A. Finding the observed information matrix when using the EM algorithm // 
Journal of the Royal Statistical Society: Series B (Methodological). – 1982. – Vol. 44. – No. 2. – P. 226–233.
\item
Ross S. M. A first course in probability. 8th ed. – Upper Saddle River, N. J.: Prentice Hall, 2010. – 640p., p. 348.
\item
Little R. J. A., Rubin D. B. Statistical analysis with missing data 2nd ed. – Hoboken, N. J.: Wiley-Interscience, Jon Wiley \& Sons, Inc., 2002. – 381p. \\
\item
Schreier P. J., Scharf L. L. Statistical Signal Processing of Complex-Valued Data: The Theory of Improper and Noncircular Signals – Cambridge: Cambridge University Press, 2010. – 309p., pp. 41-42.
\item
Nocedal J., Wright S.J. Numerical Optimization. 2nd ed. – New York: Springer, 2006. –  664p., pp. 10-37.
\end{enumerate}
\clearpage
\begin{center}
\fontsize{16}{20}\selectfont \color{teal}{\textbf{Приложение №1. Об апостериорном распределении сигналов, полученном на Е-шаге}}
\end{center}
\clearpage
\begin{center}
\fontsize{16}{20}\selectfont \color{teal}{\textbf{Приложение №2. О реализации М-шага}}
\end{center}
На практике, если $\text{A}$ соответствует антенной решетки типа ULA, удобно искать оптимальный $u=\sin(\theta)$, а затем находить $\theta$ как $\arcsin(u)$. Для удобства введем новые обозначения: $C=\Expect_{(X_m,S)|\mathcal{I}}[X S^*], D=\Expect_{(X_m,S)|\mathcal{I}}[S S^*]$. Соответственно минимизации подлежит функция
\begin{equation}
\mathcal{Q}_1(u|u^{\tau-1)}) = ||\mathbf{\Lambda}^{-1/2} (C-\widetilde{A}(u)D) ||_F^2,
\end{equation}
где
\begin{equation}
\widetilde{A}(u) = \begin{bmatrix}
1&1&\dots&1\\
e^{-2j\pi \frac{\text{d}}{\lambda}u_1}& e^{-2j\pi \frac{\text{d}}{\lambda}u_2}&\dots&e^{-2j\pi \frac{\text{d}}{\lambda}u_\text{K}}\\
\dots&\dots&\ddots&\vdots\\
e^{-2j\pi (\text{L}-1) \frac{\text{d}}{\lambda}u_1}& e^{-2j\pi (\text{L}-1) \frac{\text{d}}{\lambda}u_2}&\dots&e^{-2j\pi (\text{L}-1) \frac{\text{d}}{\lambda}u_\text{K}}\\
\end{bmatrix}.
\nonumber
\end{equation}
Для ускорения оптимизации можно оптимизировать не $\mathcal{Q}_1(u|u^{(\tau-1)})$, а суррогатную функцию $\mathcal{G}(u|u^{(\tau-1)})$, построенную так:
\begin{equation}
\mathcal{G}(u|u^{(\tau-1)}) = \mathcal{Q}_1(u^{(\tau-1)}|u^{(\tau-1)}) + \grad \mathcal{Q}_1(u^{(\tau-1)}|u^{(\tau-1)})^T (u-u^{(\tau-1)}) + \frac{1}{2} (u-u^{(\tau-1)})^T \mathbf{H} (u-u^{(\tau-1)}),
\end{equation}
где
\begin{equation}
\mathbf{H} = \begin{bmatrix}
\max(|\grad_1 \mathcal{Q}_1(u^{(\tau-1)}|u^{(\tau-1)})|,\varepsilon)&0&\dots&0\\
0& \max(|\grad_2 \mathcal{Q}_1(u^{(\tau-1)}|u^{(\tau-1)})|,\varepsilon)&\dots&0\\
\dots&\dots&\ddots&\vdots\\
0& 0&\dots& \max(|\grad_\text{K} \mathcal{Q}_1(u^{(\tau-1)}|u^{(\tau-1)})|,\varepsilon)\\
\end{bmatrix},
\nonumber
\end{equation}
причем $\grad_1 \mathcal{Q}_1(u^{(\tau-1)}|u^{(\tau-1)})$ -- $i$-я компонента градиента $\mathcal{Q}_1(u^{(\tau-1)}|u^{(\tau-1)})$.
Теперь эту суррогатную функцию надо минимизировать (или, эквивалентно, найти направление шага $\rho^{(\tau)}=u-u^{(\tau-1)}$):
\begin{equation}
\min_u \mathcal{G}(u|u^{(\tau-1)}).
\end{equation}
Подставим $u = u^{(\tau-1)}+\rho$:
\begin{equation}
\mathcal{G}(u^{(\tau-1)} + \rho| u^{(\tau-1)}) = \mathcal{Q}_1(u^{(\tau-1)}| u^{(\tau-1)}) + \grad \mathcal{Q}_1 (u^{(\tau-1)}| u^{(\tau-1)})^T\rho + \frac{1}{2}\rho^T\mathbf{H}\rho.
\end{equation}
Для нахождения минимума берем градиент по $\rho$:
\begin{equation}
\grad_\rho \mathcal{G}(u^{(\tau-1)} + \rho| u^{(\tau-1)}) =  \grad \mathcal{Q}_1(u^{(\tau-1)}| u^{(\tau-1)})   + \mathbf{H}\rho,
\end{equation}
при $\grad_\rho \mathcal{G}=0$ будет выполняться:
\begin{equation}
0 =  \grad \mathcal{Q}_1(u^{(\tau-1)}| u^{(\tau-1)})   + \mathbf{H}\rho,
\end{equation}
решаем относительно $\rho^{(\tau)}$:
\begin{equation}
\rho^{(\tau)} = -\mathbf{H}^{-1}  \grad \mathcal{Q}_1(u^{(\tau-1)}| u^{(\tau-1)}).
\end{equation}
Важно заметить, что производная по направлению $\grad \mathcal{Q}_1(u^{(\tau-1)}| u^{(\tau-1)})^T \rho^{(\tau)}$ принимает вид:
\begin{equation}
\grad \mathcal{Q}_1(u^{(\tau-1)}| u^{(\tau-1)})^T \rho^{(\tau)} = - \grad \mathcal{Q}_1(u^{(\tau-1)}| u^{(\tau-1)})^T \mathbf{H}^{-1}  \grad \mathcal{Q}_1(u^{(\tau-1)}| u^{(\tau-1)}),
\end{equation} 
и поскольку $\mathbf{H}$ -- положительно определенная матрица, производная по направлению может принимать лишь неположительные значения, а значит направление $\rho^{(\tau)}$ соответствует невозрастанию функции.
Далее используется backtracking line search для гарантии невозрастания $\mathcal{Q}_1$:
\begin{equation}
u^{(\tau)}=u^{(\tau-1)} + \alpha_m \rho^{(\tau)},
\end{equation}
где $\alpha_m$ -- первое $\alpha=\alpha_0 \beta^m, m \in \Natural$, такое что $\mathcal{Q}_1(u^{(\tau)}| u^{(\tau-1)}) \le \mathcal{Q}_1(u^{(\tau-1)}| u^{(\tau-1)})$ для фиксированных $\alpha_0 > 0, \beta \in (0;1)$.\\
\end{document}

