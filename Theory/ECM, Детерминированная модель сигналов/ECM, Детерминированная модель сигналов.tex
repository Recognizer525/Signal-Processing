\documentclass[11pt]{article}
\usepackage[english,russian]{babel}
\usepackage[utf8]{inputenc}
\usepackage[a4paper, left=2.5cm, right=1.5cm, top=2.5cm, bottom=2.5cm]{geometry}
\usepackage{animate} 
\usepackage{graphicx}
\usepackage{amsmath}
\usepackage{longtable}
\usepackage{amssymb}
\usepackage{physics}
\usepackage{tikz}
\usepackage{comment}
\usepackage{animate} 
\usepackage{graphicx}
\usepackage{amsmath}
\usepackage{longtable}
\usepackage{amssymb}
\usepackage{physics}
\usepackage{tikz}
\usepackage{comment}
\usepackage{colortbl}
\usepackage{xcolor}
\usepackage[normalem]{ulem}
\usepackage{float}
\usepackage{wrapfig}
\usepackage{cancel}
\usepackage{mathtools}
\usepackage[most]{tcolorbox}
\usepackage[mathscr]{euscript}

\DeclarePairedDelimiter\ceil{\lceil}{\rceil}
 
\newcommand{\Expect}{\mathbb{E}}
\newcommand{\Var}{\mathcal{D}}
\newcommand{\Cov}{\mathsf{cov}}
\newcommand{\Norm}{\mathcal{N}}
\newcommand{\NormComplex}{\mathcal{CN}}
\newcommand{\Real}{\mathbb{R}}
\newcommand{\Int}{\mathbb{Z}}
\newcommand{\XSig}{\mathbf{x}}
\newcommand{\Ssig}{\mathbf{s}}
\newcommand{\Nsig}{\mathbf{n}}
\newcommand{\Rs}{\mathbf{R}_s}
\newcommand{\Rn}{\mathbf{R}_n}
\newcommand{\DK}{\mathbf{D}_{KL}}
\DeclarePairedDelimiterX{\infdivx}[2]{(}{)}{%
  #1\;\delimsize\|\;#2%
}
\DeclareMathOperator*{\argmax}{argmax}
\DeclareMathOperator*{\argmin}{argmin}
\DeclareMathOperator{\Det}{Det}
\newcommand{\infdiv}{D_{KL}\infdivx}
\newcommand\Fontvi{\fontsize{8.2}{7.2}\selectfont}
\newcommand\Fontvia{\fontsize{9}{8}\selectfont}
\newcommand\Fontvib{\fontsize{10.8}{9.6}\selectfont}
\newcommand\Fontvic{\fontsize{8.0}{7.0}\selectfont}
\newcommand{\myitem}{\item[\checkmark]}
%\newcommand{\myitem}{\item[\squares]}

\begin{document}
\begin{center}
\fontsize{20}{23}\selectfont \color{red}{\textbf{ECM, Детерминированная модель сигналов}}
\end{center}
Введем некоторые условные обозначения:
\begin{itemize}
\item
$\theta$ --- вектор направлений прибытия сигнала (DoA);
\item
$\tau$ --- итерация ЕCМ-алгоритма, начальная оценка параметров $\theta$;
\item
$t$ --- момент времени (а заодно и номер кадра (snapshot));
\item
$L$ --- число датчиков;
\item
$M$ --- число источников (источники разделяют общую длину центральной волны $\chi$);
\item
$G$ --- число независимых кадров/снимков (snapshot), сделанных в разные моменты времени;
\item
$S$ --- набор детерминированных сигналов, испускаемых источниками в моменты времени $t=\overline{1,G}$, $S_t$ соответствует сигналу в момент времени $t$, представляет собой матрицу размера $G \times M$;
\item
$N$ --- набор шумов (случайная величина), связанных с датчиками в моменты времени $t=\overline{1,G}$, $N_t$ соответствует шуму в момент времени $t$, представляет собой матрицу размера $G \times L$;
\item
$n$ --- набор шумов (реализация), связанных с датчиками в моменты времени $t=\overline{1,G}$, $n_t$ соответствует шуму в момент времени $t$;
\item
$X$ --- набор сигналов (случайная величина), полученных датчиками в моменты времени $t=\overline{1,G}$, $X_t$ соответствует сигналу в момент времени $t$, представляет собой матрицу размера $G \times L$;
\item
$x$ --- набор сигналов (реализация), полученных датчиками в моменты времени $t=\overline{1,G}$, $x_t$ соответствует сигналу в момент времени $t$;
\item
$X_o$ --- наблюдаемая часть (случайная величина) $X$, $X_{o,t}$ соответствует сигналу в момент времени $t$; 
\item
$x_o$ --- наблюдаемая часть (реализация) $X$, $x_{o,t}$ соответствует сигналу в момент времени $t$;
\item
$X_m$ --- ненаблюдаемая часть (случайная величина) $X$, $X_{m,t}$ соответствует сигналу в момент времени $t$;
\item
$x_m$ --- ненаблюдаемая часть (реализация) $X$, $x_{m,t}$ соответствует сигналу в момент времени $t$;
\item
$\mathbf{O}_{D_1 \times D_2}$ --- нулевая матрица размера $D_1 \times D_2$;
\item
Полученный сигнал (итоговый сигнал, получаемый массивом датчиков):
\begin{equation}
\begin{gathered}
X_t=A(\theta)S_t+N_t,
\end{gathered}
\end{equation}
где $S_t$  -- детерминированные сигналы, $t=\overline{1,G}$, $N_t \sim CN(\mathbf{O}_{L \times 1}, \mathbf{\Lambda})$,  $\theta=[\theta_1,...,\theta_M]^T$ --- вектор направлений прибытия сигнала, $A(\theta)$ (далее -- $A$) представляет собой матрицу управляющих векторов размера $L \times M$, $\mathbf{\Lambda}$ предполагается диагональной. Рассматривается случай, когда массив антенн является линейным и равномерным.
\begin{gather}
A(\theta) = \begin{bmatrix}
1&1&\dots&1\\
e^{-2j\pi \frac{d}{\lambda}\sin(\theta_1)}& e^{-2j\pi \frac{d}{\lambda}\sin(\theta_2)}&\dots&e^{-2j\pi \frac{d}{\lambda}\sin(\theta_M)}\\
\dots&\dots&\ddots&\vdots\\
e^{-2j\pi (L-1) \frac{d}{\lambda}\sin(\theta_1)}& e^{-2j\pi (L-1) \frac{d}{\lambda}\sin(\theta_2)}&\dots&e^{-2j\pi (L-1) \frac{d}{\lambda}\sin(\theta_M)}\\
\end{bmatrix}.
\nonumber
\end{gather}
\end{itemize}
\begin{itemize}
\item
$L_{o,t}$ --- число исправных сенсоров в момент времени $t$;
\item
 $L_{m,t}$ --- число неисправных сенсоров в момент времени $t$;
\item 
$A_{o,t}$ --- матрица, образованная теми строками матрицы $A$, которые соответствуют работающим сенсорам в момент времени $t$; 
\item
$A_{m,t}$ --- матрица, образованная теми строками матрицы $A$, которые соответствуют неисправным сенсорам в момент времени $t$;
\item
$\mathbf{\Lambda}_{m,t}$ --- ковариационная матрица шума на неисправных сенсорах в момент времени $t$;
\item 
 $\mathbf{\Lambda}_{o,t}$ --- ковариационная матрица шума на исправных сенсорах в момент времени $t$.
\end{itemize}
Рассмотрим 2 случая:
\begin{itemize}
\item
Известный шум
\item
Неизвестный шум
\end{itemize}
\begin{center}
\fontsize{16}{20}\selectfont \color{teal}{\textbf{Известный шум}}
\end{center}
Воспользуемся ЕCМ-алгоритмом для того, чтобы определить значения параметров $\Psi = (\theta, S)$, пропущенные значения $X_m=\{X_{m,t}\}_{t=1}^G$ рассматриваются как латентные переменные.
\begin{center}
\fontsize{14}{18}\selectfont \color{red}{\textbf{Е-шаг}}
\end{center}
Требуется найти условное математическое ожидание с учетом апостериорного распределения ненаблюдаемых/пропущенных принятых сигналов и текущей оценки параметров
\begin{equation}
 \Expect_{X_m|X_o, \Psi^{(\tau)}}[\log P(X_o, X_m)]
\end{equation}
Сначала найдем апостериорное распределение $P(X_m|X_o=x_o,\Psi)$, воспользуемся формулой Байеса:
\begin{gather}
P(X_m|X_o=x_o,\Psi) = \frac{P(X_o, X_m|\Psi)}{P(X_o|\Psi)} = \frac{P(X|\Psi)}{P(X_o|\Psi)}
\end{gather}
\begin{gather*}
X_t = AS_t + N_t \\
X_t \sim CN(A S_t,\mathbf{\Lambda})\\
X_{o,t} \sim CN( A_{o,t}S_t, \mathbf{\Lambda_{o,t}})\\
\end{gather*}
\begin{gather}
P(X|\Psi) = \prod_{t=1}^G \frac{1}{\pi^L \Det(\mathbf{\Lambda})}e^{-(X_t-AS_t)^H (\mathbf{\Lambda})^{-1}(X_t-AS_t)},
\end{gather}
\begin{gather}
P(X_o|\Psi) = \prod_{t=1}^G \frac{1}{\pi^{L_{o,t}} \Det(\mathbf{\Lambda}_{o,t})}e^{-(X_{o,t}-A_{o,t}S_t)^H (\mathbf{\Lambda}_{o,t})^{-1}(X_{o,t}-A_{o,t}S_t)},
\end{gather}
Параметры апостериорного распределения $P(X_m|X_o=x_o,\Psi)$ можно найти исходя из следующих формул:
\begin{equation}
\left\{ \begin{gathered} 
m_{x_{m,t}|x_{o,t}} = m_{x_{m,t}} + K_{x_{m,t},x_{o,t}}K_{x_{o,t},x_{o,t}}^{-1}\cdot(x_{o,t}-m_{x_{o,t}}) \\
K_{x_{m,t}|x_{o,t}} = K_{x_{m,t},x_{m,t}}-K_{x_{m,t},x_{o,t}}K_{x_{o,t},x_{o,t}}^{-1}K_{x_{o,t},x_{m,t}},
\end{gathered} \right.
\end{equation}
В рамках данной задачи:
\begin{equation}
\left\{ \begin{gathered} 
K_{x_{o,t},x_{o,t}} = \mathbf{\Lambda}_{o,t} \\
K_{x_{o,t},x_{m,t}} = \hat{K}_{x_{o,t},x_{m,t}} \\
K_{x_{m,t},x_{o,t}} = \hat{K}^H_{x_{o,t},x_{m,t}} \\
K_{x_{m,t},x_{m,t}} = \mathbf{\Lambda}_{m,t} \\
m_{x_{o,t}}=A_{o,t}S_t \\
m_{x_{m,t}}=A_{m,t}S_t
\end{gathered} \right.
\end{equation}
где $\hat{K}_{x_{o,t},x_{m,t}}$ -- выборочная оценка ковариации на основе полных наблюдений, т.е. таких, что $x_t = x_{o,t}$
\begin{equation}
\left\{ \begin{gathered} 
m_{x_{m,t}|x_{o,t}} = A_{m,t}S_t + A_{m,t}\mathbf{P} A_{o,t}^H(\mathbf{\Lambda}_{o,t})^{-1}\cdot(x_o-A_{o,t}S_t) \\
K_{x_{m,t}|x_{o,t}} = \mathbf{\Lambda}_{m,t}-A_{m,t}\mathbf{P} A_{o,t}^H(\mathbf{\Lambda}_{o,t})^{-1}A_{o,t}\mathbf{P} A_{m,t}^H,
\end{gathered} \right.
\end{equation}
Вернемся к ранее рассмотренному условному математическому ожиданию:
\begin{equation*}
 \Expect_{X_m|X_o, \Psi^{(\tau)}}[\log P(X_o, X_m)].
\end{equation*}
Его следует максимизировать, мы можем перейти от логарифма произведения к сумме логарифмов. Определим, как будет выглядеть это УМО для произвольно выбранного элемента выборки $X_t$:
\begin{equation*}
 \Expect_{X_{m,t}|X_{o,t}, \Psi^{(\tau)}}[\log P(X_{o,t}, X_{m,t})] = \\
\end{equation*}
\begin{equation}
\begin{gathered}
 \log \left(\frac{1}{\pi^{L}\Det(\Sigma)}e^{-(X_t-\mu_{X_t})^H\Sigma^{-1}(X_t-\mu_{X_t})}\right) = \\
-L\log(\pi) - \log(\Det(\Sigma)) -(X_t-\mu_{X_t})^H\Sigma^{-1}(X_t-\mu_{X_t})= \\
\end{gathered}
\end{equation}
\begin{align*}
& \Expect_{X_m|X_o=x_o, \Psi^{(\tau)}} \Big[ \begin{bmatrix} X_{m, t} \\  X_{o, t} \end{bmatrix}^H \begin{bmatrix} (\Sigma^{-1})_{x_m, x_m} & (\Sigma^{-1})_{x_m, x_o} \\ (\Sigma^{-1})_{x_o, x_m} & (\Sigma^{-1})_{x_o, x_o} \end{bmatrix} \begin{bmatrix} X_{m, t} \\  X_{o, t} \end{bmatrix} \Big] \\
=&\hspace{16pt} \Expect_{X_m|X_o=x_o, \Psi^{(\tau)}} \Big[ ( X_{m,t})^H (\Sigma^{-1})_{x_m, x_m} ( X_{m, t}) \Big] \\ 
&+ 2 \Expect_{X_m|X_o=x_o, \Psi^{(\tau)}} \Big[ ( X_{o, t})^H (\Sigma^{-1})_{x_o, x_m} ( X_{m, t}) \Big] \\ 
&+ \hspace{5pt} \Expect_{X_m|X_o=x_o, \Psi^{(\tau)}} \Big[ ( X_{o, t})^H (\Sigma^{-1})_{x_o, x_o} (  X_{o, t}) \Big] \\
=&\hspace{16pt} \Expect_{X_m|X_o=x_o, \Psi^{(\tau)}} \Big[ ( X_{m, t})^H (\Sigma^{-1})_{x_m, x_m} ( X_{o, t}) \Big] \\ 
&+ 2 ( x_{o, t})^H (\Sigma^{-1})_{x_o, x_m} \Expect_{X_m|X_o=x_o, \Psi^{(\tau)}} \Big[  X_{m, t} \Big] \\ 
&+ \hspace{5pt} x_{o, t}^H (\Sigma^{-1})_{x_o, x_o} x_{o, t}
\end{align*}
\begin{center}
\fontsize{14}{18}\selectfont \color{red}{\textbf{M-шаг}}
\end{center}
Требуется найти наилучшую оценку параметров, решив следующую задачу оптимизации:
\begin{equation*}
\begin{gathered}
\Psi^{(\tau+1)}=\argmax_{\Psi} \Expect_{X_m|X_o=x_o, \Psi^{(\tau)}}[\log P(X_o, X_m)] =\\
\argmax_{\Psi}  \Expect_{X_m|X_o=x_o, \Psi^{(\tau)}}\left[\sum_{t=1}^G\log P(X_{o,t}, X_{m,t})\right] = \\
\argmax_{\Psi}  \Expect_{X_m|X_o=x_o, \Psi^{(\tau)}}\left[\sum_{t=1}^G \log \left(\frac{1}{\pi^{L}\Det(\Sigma)}e^{-(X_t-\mu_{X_t})^H\Sigma^{-1}(X_t-\mu_{X_t})}\right)\right]
\end{gathered}
\end{equation*}
\begin{center}
\fontsize{14}{18}\selectfont \color{red}{\textbf{Первый СM-шаг}}
\end{center}
\begin{center}
\fontsize{14}{18}\selectfont \color{red}{\textbf{Второй СM-шаг}}
\end{center}
\begin{center}
\fontsize{16}{20}\selectfont \color{teal}{\textbf{Неизвестный шум}}
\end{center}
\begin{center}
\fontsize{14}{18}\selectfont \color{red}{\textbf{Е-шаг}}
\end{center}
\begin{center}
\fontsize{14}{18}\selectfont \color{red}{\textbf{M-шаг}}
\end{center}
\begin{center}
\fontsize{14}{18}\selectfont \color{red}{\textbf{Первый СM-шаг}}
\end{center}
\begin{center}
\fontsize{14}{18}\selectfont \color{red}{\textbf{Второй СM-шаг}}
\end{center}
\begin{center}
\fontsize{16}{20}\selectfont \color{teal}{\textbf{Список источников}}
\end{center}
\end{document}
