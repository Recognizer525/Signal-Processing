\documentclass[11pt]{article}
\usepackage[english,russian]{babel}
\usepackage[utf8]{inputenc}
\usepackage[a4paper, left=2.5cm, right=1.5cm, top=2.5cm, bottom=2.5cm]{geometry}
\usepackage{animate} 
\usepackage{graphicx}
\usepackage{amsmath}
\usepackage{longtable}
\usepackage{amssymb}
\usepackage{physics}
\usepackage{tikz}
\usepackage{comment}
\usepackage{animate} 
\usepackage{graphicx}
\usepackage{amsmath}
\usepackage{longtable}
\usepackage{amssymb}
\usepackage{physics}
\usepackage{tikz}
\usepackage{comment}
\usepackage{colortbl}
%\usepackage{xcolor}
\usepackage[normalem]{ulem}
\usepackage{float}
\usepackage{wrapfig}
\usepackage{cancel}
\usepackage{mathtools}
\usepackage[most]{tcolorbox}
\usepackage[mathscr]{euscript}

\usepackage[dvipsnames]{xcolor}

\DeclarePairedDelimiter\ceil{\lceil}{\rceil}
 
\newcommand{\Expect}{\mathbb{E}}
\newcommand{\Var}{\mathcal{D}}
\newcommand{\Cov}{\mathsf{cov}}
\newcommand{\Norm}{\mathcal{N}}
\newcommand{\NormComplex}{\mathcal{CN}}
\newcommand{\Natural}{\mathbb{N}}
\newcommand{\Real}{\mathbb{R}}
\newcommand{\Complex}{\mathbb{C}}
\newcommand{\Int}{\mathbb{Z}}
\newcommand{\DK}{\mathbf{D}_{KL}}
\DeclarePairedDelimiterX{\infdivx}[2]{(}{)}{%
  #1\;\delimsize\|\;#2%
}
\DeclareMathOperator*{\argmax}{arg\,max}
\DeclareMathOperator*{\argmin}{arg\,min}
\DeclareMathOperator{\Det}{Det}
\newcommand{\infdiv}{D_{KL}\infdivx}
\newcommand\Fontvi{\fontsize{8.2}{7.2}\selectfont}
\newcommand\Fontvia{\fontsize{9}{8}\selectfont}
\newcommand\Fontvib{\fontsize{10.8}{9.6}\selectfont}
\newcommand\Fontvic{\fontsize{8.0}{7.0}\selectfont}
\newcommand{\myitem}{\item[\checkmark]}
%\newcommand{\myitem}{\item[\squares]}

\begin{document}
\begin{center}
\fontsize{20}{23}\selectfont \color{red}{\textbf{Список источников с пояснениями}}
\end{center}
\begin{enumerate}
\item
Dempster A. P., Laird N. M., Rubin D. B. Maximum likelihood from incomplete data via the EM algorithm // Journal of the Royal Statistical Society: Series B (Methodological). – 1977. – Vol. 39. – No. 1.  – P. 1–38. \\
Истоки ЕМ-алгоритма
\item
Wu C. F. J. On the convergence properties of the EM algorithm // The Annals of Statistics. – 1983. – Vol. 11. –  No. 1. –  P. 95–103. \\
Доказательство корректности Generalized EM
\item
Louis T. A. Finding the observed information matrix when using the EM algorithm // 
Journal of the Royal Statistical Society: Series B (Methodological). – 1982. – Vol. 44. – No. 2. – P. 226–233. \\
Идея для Е-шага в моем случае
\item
Ross S. M. A first course in probability. 8th ed. – Upper Saddle River, N. J.: Prentice Hall, 2010. – 640p., p. 348. \\
Закон полной дисперсии
\item
Little R. J. A., Rubin D. B. Statistical analysis with missing data. 2nd ed. – Hoboken, N. J.: Wiley-Interscience, Jon Wiley \& Sons, Inc., 2002. – 381p. \\
Применение ЕМ-алгоритма для оценки пропусков (книгу нужно скачать, есть и 3-е издание)
\item
Schreier P. J., Scharf L. L. Statistical Signal Processing of Complex-Valued Data: The Theory of Improper and Noncircular Signals – Cambridge: Cambridge University Press, 2010. – 309p., pp. 41-42. \\
Условное распределение комплексного нормального распределения.
\item
Nocedal J., Wright S.J. Numerical Optimization. 2nd ed. – New York: Springer, 2006. –  664p., pp. 10-37. \\
Важность Backtracking line search и знака производной по направлению.
\end{enumerate}
\end{document}

