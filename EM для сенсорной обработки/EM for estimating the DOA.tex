\documentclass[11pt]{article}
\usepackage[english,russian]{babel}
\usepackage[utf8]{inputenc}
\usepackage[a4paper, left=2.5cm, right=1.5cm, top=2.5cm, bottom=2.5cm]{geometry}
\usepackage{animate} 
\usepackage{graphicx}
\usepackage{amsmath}
\usepackage{longtable}
\usepackage{amssymb}
\usepackage{physics}
\usepackage{tikz}
\usepackage{comment}
\usepackage{animate} 
\usepackage{graphicx}
\usepackage{amsmath}
\usepackage{longtable}
\usepackage{amssymb}
\usepackage{physics}
\usepackage{tikz}
\usepackage{comment}
\usepackage{colortbl}
\usepackage{xcolor}
\usepackage[normalem]{ulem}
\usepackage{float}
\usepackage{wrapfig}
\usepackage{cancel}
\usepackage{mathtools}
\usepackage[most]{tcolorbox}
\usepackage[mathscr]{euscript}

\DeclarePairedDelimiter\ceil{\lceil}{\rceil}
 
\newcommand{\Expect}{\mathsf{M}}
\newcommand{\Var}{\mathsf{D}}
\newcommand{\Cov}{\mathsf{cov}}
\newcommand{\Norm}{\mathcal{N}}
\newcommand{\NormComplex}{\mathcal{CN}}
\newcommand{\Real}{\mathbb{R}}
\newcommand{\Int}{\mathbb{Z}}
\newcommand{\XSig}{\mathbf{x}}
\newcommand{\Ssig}{\mathbf{s}}
\newcommand{\Nsig}{\mathbf{n}}
\newcommand{\Rs}{\mathbf{R}_s}
\newcommand{\Rn}{\mathbf{R}_n}
\newcommand{\DK}{\mathbf{D}_{KL}}
\DeclarePairedDelimiterX{\infdivx}[2]{(}{)}{%
  #1\;\delimsize\|\;#2%
}
\DeclareMathOperator*{\argmax}{arg\,max}
\DeclareMathOperator*{\argmin}{arg\,min}
\newcommand{\infdiv}{D_{KL}\infdivx}
\newcommand\Fontvi{\fontsize{8.2}{7.2}\selectfont}
\newcommand\Fontvia{\fontsize{9}{8}\selectfont}
\newcommand\Fontvib{\fontsize{10.8}{9.6}\selectfont}
\newcommand\Fontvic{\fontsize{8.0}{7.0}\selectfont}
\newcommand{\myitem}{\item[\checkmark]}
%\newcommand{\myitem}{\item[\squares]}

\begin{document}
\begin{center}
\fontsize{20}{23}\selectfont \color{red}{\textbf{ЕМ-алгоритм для оценки направления прибытия сигнала}}
\end{center}
Введем некоторые условные обозначения:
\begin{itemize}
\item
$\theta$ -- вектор направлений прибытия сигнала (DOA);
\item
$\tau$ -- итерация ЕМ-алгоритма, начальная оценка параметров $\theta$;
\item
$t$ -- момент времени (а заодно и номер кадра (snapshot));
\item
$L$ -- число датчиков;
\item
$M$ -- число источников (источники разделяют общую длину центральной волны $\chi$);
\item
$G$ -- число независимых кадров/снимков (snapshot), сделанных в разные моменты времени;
\item
$S$ -- набор сигналов (случайная величина), испускаемых источниками в моменты времени $t=\overline{1,G}$, $S_t$ соответствует сигналу в момент времени $t$;
\item
$N$ -- набор шумов (случайная величина), связанных с датчиками в моменты времени $t=\overline{1,G}$, $N_t$ соответствует шуму в момент времени $t$;
\item
$X$ -- набор сигналов (случайная величина), полученных датчиками в моменты времени $t=\overline{1,G}$, $X_t$ соответствует сигналу в момент времени $t$;
\item
$s$ -- набор сигналов (реализация), испускаемых источниками в моменты времени $t=\overline{1,G}$, $s_t$ соответствует сигналу в момент времени $t$;
\item
$n$ -- набор шумов (реализация), связанных с датчиками в моменты времени $t=\overline{1,G}$, $n_t$ соответствует шуму в момент времени $t$;
\item
$x$ -- набор сигналов (реализация), полученных датчиками в моменты времени $t=\overline{1,G}$, $x_t$ соответствует сигналу в момент времени $t$;
\item
$O_{D_1 \times D_2}$ -- нулевая матрица размера $D_1 \times D_2$;
\item
Итоговый сигнал, получаемый массивом датчиков:
\begin{equation}
\begin{gathered}
X_t=A(\theta)S_t+N_t,
\end{gathered}
\end{equation}
где $S_t \sim CN(0,\Gamma_s),t=\overline{1,G}$, $N_t \sim CN(0,\Gamma_n), t=\overline{1,G}$, $S_t$ имеет размер $M \times 1$,  $N_t$ имеет размер $L \times 1$, $\theta=[\theta_1,...,\theta_M]$ -- вектор направлений прибытия сигнала, $A(\theta)$ (далее - $A$) представляет собой матрицу управляющих векторов размера $L \times M$, $\Gamma_s$ и $\Gamma_n$ предполагаются диагольными.
\begin{gather}
A(\theta) = \begin{bmatrix}
1&1&\dots&1\\
e^{-2j\pi \frac{d}{\lambda}sin(\theta_1)}& e^{-2j\pi \frac{d}{\lambda}sin(\theta_2)}&\dots&e^{-2j\pi \frac{d}{\lambda}sin(\theta_M)}\\
\dots&\dots&\ddots&\vdots\\
e^{-2j\pi (L-1) \frac{d}{\lambda}sin(\theta_1)}& e^{-2j\pi (L-1) \frac{d}{\lambda}sin(\theta_2)}&\dots&e^{-2j\pi (L-1) \frac{d}{\lambda}sin(\theta_M)}\\
\end{bmatrix}.
\nonumber
\end{gather}
\end{itemize}
Воспользуемся ЕМ-алгоритмом для того, чтобы определить значения параметров $\theta$, значения сигналов $S_t, t=\overline{1,G}$ рассматриваются как латентные переменные. 
Пусть $X$, $S$ и $N$ набор итоговых сигналов полученных $L$ датчиками за моменты времени $t=\overline{1,G}$ и набор выпущенных $M$ источниками сигналов и набор шумов за моменты времени $t=\overline{1,G}$, соответственно. $X$, $S$ и $N$ представляют из себя матрицы размеров $G \times L$, $G \times M$ и $G \times L$ соответственно.
\begin{center}
\fontsize{16}{20}\selectfont \color{red}{\textbf{Е-шаг}}
\end{center}
Требуется найти апостериорное распределение $P(S|X=x,\theta)$, воспользуемся формулой Байеса:
\begin{gather}
P(S|X=x,\theta) = \frac{P(X|S,\theta)P(S|\theta)}{P(X|\theta)}
\end{gather}
\begin{gather}
P(S|\theta) = \prod_{t=1}^G \frac{1}{\pi^M |\Gamma_s|}e^{-S_t^H\Gamma_s^{-1}S_t},
\end{gather}
\begin{gather*}
X_t = AS_t + N_t \\
X_t \sim CN(0, A\Gamma_s A^H + \Gamma_n)
\end{gather*}
\begin{gather}
P(X|\theta) = \prod_{t=1}^G \frac{1}{\pi^L |A\Gamma_s  A^H + \Gamma_n|}e^{-X_t^H (A\Gamma_s A^H + \Gamma_n)^{-1}X_t},
\end{gather}
Теперь следует определиться с тем, каким будет условное распределение $P(X|S=s, \theta)$
\begin{gather*}
X_t|S_t=s_t, \theta \, \sim CN(A s_t, \Gamma_n)
\end{gather*}
\begin{gather}
P(X|S=s,\theta) = \prod_{t=1}^G \frac{1}{\pi^L |\Gamma_n|}e^{-(X_t-A s_t)^H \Gamma_n^{-1}(X_t-A s_t)},
\end{gather}
\begin{gather}
P(S|X=x,\theta) = \prod_{t=1}^G \frac{1}{\pi^M |\Gamma_n||\Gamma_s||(A\Gamma_s A^H + \Gamma_n)^{-1}|}e^{-(x_t-A\cdot S_t)^H \Gamma_n^{-1}(x_t-A\cdot S_t)-S_t^H\Gamma_s^{-1}S_t+x_t^H |(A\Gamma_s A^H + \Gamma_n)^{-1}|^{-1}x_t},
\end{gather}
Таким образом:
\begin{gather}
P(S|X=x,\theta) = \prod_{t=1}^G \frac{1}{\pi^M |K_{S_t|X_t}|}e^{-(S_t-\mu_{S_t|X_t})^H K_{S_t|X_t}^{-1} (S_t-\mu_{S_t|X_t})}
\end{gather}
Условное распределение также будет комплексным гауссовским, а его параметры будут определяться в соответствии с модифицированной теоремой о нормальной корреляции (см. файл Conditional\_Distribution.pdf).
Определим кросс-ковариацию между $S$ и $X$.
\begin{equation}
\begin{gathered}
Cov(S,X)=E[SX^H]=E[S(AS+N)^H]=E[S(AS)^H]+E[SN^H]=E[S(AS)^H]=\\E[SS^HA^H]=E[SS^H]A^H=\Gamma_sA^H
\end{gathered}
\end{equation}
Данные переходы возможны ввиду свойств ковариации и независимости $N$ и $S$. $Cov(X,S)=(Cov(S,X))^H$:
\begin{gather}
Cov(X,S) = A\Gamma_s
\end{gather}
\begin{gather}
\mu = \begin{pmatrix}
O_{M\times1}\\
O_{L\times1}
\end{pmatrix},
\Sigma = \begin{pmatrix}
\Gamma_s&\Gamma_sA^H\\
A\Gamma_s&A\Gamma_sA^H+\Gamma_n
\end{pmatrix}.
\end{gather}
В соответствии с выкладками, полученными в файле  Conditional\_Distribution.pdf, параметры апостериорной плотности будут определяться по следующим формулам:
\begin{equation}
\left\{ \begin{gathered} 
K_{S_t|X_t} = \Sigma_{S_t,S_t}-\Sigma_{S_t,X_t}\Sigma_{X_t,X_t}^{-1}\Sigma_{X_t,S_t} \\
\mu_{S_t|X_t} = \mu_{S_t} + \Sigma_{S_t,X_t}\Sigma_{X_t,X_t}^{-1}(x_t-\mu_{X_t})
\end{gathered} \right.
\end{equation}
Подствим в эту формулу полученные нами значения:
\begin{equation}
\left\{ \begin{gathered} 
K_{S_t|X_t} = \Gamma_s-\Gamma_sA^H(A\Gamma_sA^H+\Gamma_n)^{-1}A\Gamma_s \\
\mu_{S_t|X_t} = \Gamma_sA^H(A\Gamma_sA^H+\Gamma_n)^{-1} x_t
\end{gathered} \right.
\end{equation}
\begin{center}
\fontsize{16}{20}\selectfont \color{red}{\textbf{M-шаг}}
\end{center}
Требуется найти наилучшую оценку параметров, решив следующую задачу оптимизации:
\begin{gather}
\theta^{(\tau+1)}=\argmax_{\theta} E[\log P(X, \, S|\theta^{(\tau)})|X, \theta^{(\tau)}]
\end{gather}
Обозначим приведенное выше условное математическое ожидание через $Q_t$. 
Заметим, что \\ $\log P(X, \, S|\theta^{(\tau)}) = \log P(X|S,\theta^{(\tau)})P(S|\theta^{(\tau)}) = \log P(S|X,\theta^{(\tau)})P(X|\theta^{(\tau)})$.\\
Можно заметить, что $P(X|S, S) = P(X, S)$. Работать с плотостью $P(X|S, S)$ удобнее: кросс-ковариция между $S$ и $X|S$ будет представлять из себя нулевую матрицу.
Найдем совместную плотность $P(X|S, \, S|\theta^{(\tau)})$:
\begin{gather}
P(X|S, \, S|\theta^{(\tau)}) = \prod_{t=1}^G \frac{1}{\pi^{M+L}|\Sigma|}e^{-(Z_t-\mu_{Z_t})^H\Sigma^{-1}(Z_t-\mu_{Z_t})},
\end{gather}
где:
\begin{equation*}
Z_t = \begin{pmatrix}
S_t\\
X_t|S_t
\end{pmatrix},
\mu_{Z_t} = \begin{pmatrix}
\mu_{S_t}\\
\mu_{X_t|S_t}
\end{pmatrix}\\,
\Sigma = 
\begin{pmatrix}
\Sigma_{S_t,S_t}&O_{M\times L}\\
O_{L\times M}&\Sigma_{X_t|S_t,X_t|S_t}
\end{pmatrix}
\end{equation*}
Найдем логарифм совместной плотности (т.е. полное правдоподобие) $\log P(X|S, \, S|\theta^{(\tau)})$:
\begin{equation*}
\log P(X|S, \, S|\theta^{(\tau)}) = \sum_{t=1}^G \left(-(M+L)\log(\pi)-\log|\Sigma|-(Z_t-\mu_{Z_t})^H\Sigma^{-1}(Z_t-\mu_{Z_t})\right)
\end{equation*}
Теперь попробуем раскрыть УМО полного правдоподобия для одного наблюдения, с учетом полученных сигналов и текущей оценки DOA:
\begin{gather}
P(X_t|S_t, \, S_t|\theta^{(\tau)}) = \frac{1}{\pi^{M+L}|\Sigma|}e^{-(Z_t-\mu_{Z_t})^H\Sigma^{-1}(Z_t-\mu_{Z_t})}
\end{gather}
\begin{gather*}
Q_t = E[\log P(X_t|S_t, \, S_t|\theta^{(\tau)})|X_t, \theta^{(\tau)}] \\ = E[-(M+L)\log(\pi)-\log|\Sigma|-(Z_t-\mu_{Z_t})^H\Sigma^{-1}(Z_t-\mu_{Z_t})|X_t=x_t, \theta^{(\tau)}] = \\
=-(M+L)\log(\pi)-\log|\Sigma| - E[(Z_t-\mu_{Z_t})^H\Sigma^{-1}(Z_t-\mu_{Z_t})|X_t=x_t, \theta^{(\tau)}] 
=-(M+L)\log(\pi)-\log|\Sigma| \\ -   E[(X_t|S_t-\mu_{X_t|S_t})^H\Sigma_{X_t|S_t,X_t|S_t}^{-1}(X_t|S_t-\mu_{X|S})|X_t=x_t, \theta^{(\tau)}] 
- E[(S_t-\mu_{S_t})^H\Sigma_{S_t,S_t}^{-1}(S_t-\mu_{S_t})|X_t=x_t, \theta^{(\tau)}] \\
=-(M+L)\log(\pi)-\log|\Sigma| - (x_t-\mu_{X|S})^H\Sigma_{X_t|S_t,X_t|S_t}^{-1}(x_t-\mu_{X_t|S_t})\\ - E[(S_t-\mu_{s_t})^H\Sigma_{S_t,S_t}^{-1}(S_t-\mu_{S_t})|X_t=x_t,\theta^{(\tau)}] 
\end{gather*}
\begin{equation}
\begin{gathered}
Q_t = -(M+L)\log(\pi)-\log|\Sigma| - (x_t-\mu_{X_t|S_t})^H\Sigma_{X_t|S_t,X_t|S_t}^{-1}(x_t-\mu_{X_t|S_t}) \\ - E[(S_t-\mu_{S_t})^H\Sigma_{S_t,S_t}^{-1}(S_t-\mu_{S_t})|X_t=x_t,\theta^{(\tau)}] 
\end{gathered}
\end{equation}
Заметим, что первые два слагаемые, составляющие $Q_t$, не зависят от $\theta$, соответственно требуемый $argmax [\cdot]$ можно найти без их учета.
\begin{gather*}
S_t|(X_t=x_t, \theta^{(\tau)}) \sim CN(\mu_{S_t|X_t}, K_{S_t|X_t}) \Rightarrow S_t-\mu_{S_t}|(X_t=x_t, \theta^{(\tau)}) \sim CN(\mu_{S_t|X_t}-\mu_{S_t}, K_{S_t|X_t})
\end{gather*}
\begin{gather*}
\Rightarrow [\Sigma_{S_t,S_t}^{-1}]^{\frac{1}{2}}( S_t-\mu_{S_t})|(X_t=x_t, \theta^{(\tau)}) \sim CN([\Sigma_{S_t,S_t}^{-1}]^{\frac{1}{2}}(\mu_{S_t|X_t}-\mu_{S_t}),[\Sigma_{S_t,S_t}^{-1}]^{\frac{1}{2}} K_{S_t|X_t}([\Sigma_{S_t,S_t}^{-1}]^{\frac{1}{2}})^H)
\end{gather*}
Учтем, что для комплексных векторов $Y$ выполняется следующее соотношение: 
\begin{gather}
E[YY^H]=E[Y]E[Y^H]+K_{YY}.
\end{gather}
\begin{gather*}
 E[(S_t-\mu_{S_t})^H\Sigma_{S_t,S_t}^{-1}(S_t-\mu_{S_t})|X_t=x_t,\theta^{(\tau)}]  =  E[[[\Sigma_{S_t,S_t}^{-1}]^{\frac{1}{2}}(S_t-\mu_{S_t})]^H[\Sigma_{S_t,S_t}^{-1}]^{\frac{1}{2}}(S_t-\mu_{S_t})|X_t=x_t,\theta^{(\tau)}] =\\
= tr(E[[\Sigma_{S_t,S_t}^{-1}]^{\frac{1}{2}}(S_t-\mu_{S_t})[[\Sigma_{S_t,S_t}^{-1}]^{\frac{1}{2}}(S_t-\mu_{S_t})]^H]|X_t=x_t,\theta^{(\tau)})=\\
= tr([\Sigma_{S_t,S_t}^{-1}]^{\frac{1}{2}} K_{S_t|X_t}([\Sigma_{S_t,S_t}^{-1}]^{\frac{1}{2}})^H) + 
 tr([[\Sigma_{S_t,S_t}^{-1}]^{\frac{1}{2}}(\mu_{S_t|X_t}-\mu_{S_t})][[\Sigma_{S_t,S_t}^{-1}]^{\frac{1}{2}}(\mu_{S_t|X_t}-\mu_{S_t}]^H) = \\
= tr([\Sigma_{S_t,S_t}^{-1}]K_{S_t|X_t}) + (\mu_{S_t|X_t}-\mu_{S_t})^H\Sigma_{S_t,S_t}^{-1}(\mu_{S_t|X_t}-\mu_{S_t})
\end{gather*}
\begin{equation}
\begin{gathered}
Q_t = -(M+L)\log(\pi)-\log|\Sigma| - (x_t-\mu_{X_t|S_t})^H\Sigma_{X_t|S_t,X_t|S_t}^{-1}(x_t-\mu_{X_t|S_t}) \\ -  tr([\Sigma_{S_t,S_t}^{-1}]K_{S_t|X_t}) - (\mu_{S_t|X_t}-\mu_{S_t})^H\Sigma_{S_t,S_t}^{-1}(\mu_{S_t|X_t}-\mu_{S_t}) 
\end{gathered}
\end{equation}
\begin{equation}
\begin{gathered}
 E[\log P(X, \, S|\theta^{(\tau)})|X, \theta^{(\tau)}] = \sum_{t=1}^G Q_t = \sum_{t=1}^G \left(-(M+L)\log(\pi)-\log|\Sigma| - \right. \\ \left. (x_t-\mu_{X_t|S_t})^H\Sigma_{X_t|S_t,X_t|S_t}^{-1}(x_t-\mu_{X_t|S_t})  -  tr([\Sigma_{S_t,S_t}^{-1}]K_{S_t|X_t}) - (\mu_{S_t|X_t}-\mu_{S_t})^H\Sigma_{S_t,S_t}^{-1}(\mu_{S_t|X_t}-\mu_{S_t})\right)
\end{gathered}
\end{equation}
Как было сказано ранее, первые два слагаемые, составляющие $Q_t$ не зависят от $\theta$, а значит задача о поиске $\argmax_{\theta} E[\log P(X, \, S|\theta^{(\tau)})|X, \theta^{(\tau)}] $ сводится к поиску
\begin{equation}
\begin{gathered}
\argmin_{\theta}  \sum_{t=1}^G \left((x_t-\mu_{X_t|S_t})^H\Sigma_{X_t|S_t,X_t|S_t}^{-1}(x_t-\mu_{X_t|S_t}) +  tr([\Sigma_{S_t,S_t}^{-1}]K_{S_t|X_t}) \right. \\ \left. +  (\mu_{S_t|X_t}-\mu_{S_t})^H\Sigma_{S_t,S_t}^{-1}(\mu_{S_t|X_t}-\mu_{S_t})\right).
\end{gathered}
\end{equation}
Учтем, что $\mu_{S_t} = O_{M \times 1}$, $\mu_{X_t|S_t}=As_t$, в рамках задачи $s_t$ -- скрытая переменная, она оценивается так: $\hat{s}_t = \mu_{S_t|X_t}$.
\begin{equation}
\begin{gathered}
\argmin_{\theta}  \sum_{t=1}^G \left((x_t-\mu_{X_t|S_t})^H\Sigma_{X_t|S_t,X_t|S_t}^{-1}(x_t-\mu_{X_t|S_t}) +  tr([\Sigma_{S_t,S_t}^{-1}]K_{S_t|X_t}) \right. \\ \left. +  (\mu_{S_t|X_t}-\mu_{S_t})^H\Sigma_{S_t,S_t}^{-1}(\mu_{S_t|X_t}-\mu_{S_t})\right) = \\
\argmin_{\theta}  \sum_{t=1}^G \left((x_t-A\mu_{S_t|X_t})^H\Gamma_n^{-1}(x_t-A\mu_{S_t|X_t}) +  tr([\Gamma_s^{-1}]K_{S_t|X_t}) \right. \\ \left. +  (\mu_{S_t|X_t}-O_{M \times 1})^H\Gamma_s^{-1}(\mu_{S_t|X_t}-O_{M \times 1})\right) = \\
\argmin_{\theta}  \sum_{t=1}^G \left(x_t^H\Gamma_n^{-1}x_t -(A\mu_{S_t|X_t})^H\Gamma_n^{-1}x_t -x_t^H\Gamma_n^{-1}A\mu_{S_t|X_t} + (A\mu_{S_t|X_t})^H\Gamma_n^{-1}A\mu_{S_t|X_t} \right) =\\
\argmin_{\theta}  \sum_{t=1}^G \left(-(A\mu_{S_t|X_t})^H\Gamma_n^{-1}x_t -x_t^H\Gamma_n^{-1}A\mu_{S_t|X_t} + (A\mu_{S_t|X_t})^H\Gamma_n^{-1}A\mu_{S_t|X_t} \right) =\\
\argmin_{\theta}  \sum_{t=1}^G \left(-\mu_{S_t|X_t}^HA^H\Gamma_n^{-1}x_t -x_t^H\Gamma_n^{-1}A\mu_{S_t|X_t} + \mu_{S_t|X_t}^HA^H\Gamma_n^{-1}A\mu_{S_t|X_t} \right)
\end{gathered}
\end{equation}
Теперь стоит подумать о том, как вычислить минимум для этой функции. Для начала определим первую производную для минимизируемой функции.
Обозначим выражение, для которого мы ищем argmin, через $\tilde{Q}$. 
\begin{equation}
\begin{gathered}
\frac{\partial \tilde{Q}}{\partial \theta_i} = 
\sum_{t=1}^G \frac{\partial}{\partial \theta_i} \left(-x_t^H\Gamma_n^{-1}A\mu_{S_t|X_t}-\mu_{S_t|X_t}^HA^H\Gamma_n^{-1}x_t  + \mu_{S_t|X_t}^HA^H\Gamma_n^{-1}A\mu_{S_t|X_t} \right) = \\
\sum_{t=1}^G \left(-x_t^H\Gamma_n^{-1}\left(\frac{\partial A}{\partial \theta_i}\right)\mu_{S_t|X_t}-\mu_{S_t|X_t}^H\left(\frac{\partial A}{\partial \theta_i}\right)^H\Gamma_n^{-1}x_t  + \mu_{S_t|X_t}^H\left(\frac{\partial A}{\partial \theta_i}\right)^H\Gamma_n^{-1}A\mu_{S_t|X_t}  \right. \\
+ \left. \mu_{S_t|X_t}^HA^H\Gamma_n^{-1}\left(\frac{\partial A}{\partial \theta_i}\right)\mu_{S_t|X_t} \right) 
\end{gathered}
\end{equation}
\\Теперь о том, что из себя представляет производная для матрицы $A$:
\begin{gather*}
A = \begin{bmatrix}
1&\dots&1&\dots&1\\
e^{-2j\pi \frac{d}{\lambda}\sin(\theta_1)}&\dots&e^{-2j\pi \frac{d}{\lambda}\sin(\theta_i)}&\dots&e^{-2j\pi \frac{d}{\lambda}\sin(\theta_M)}\\
\dots&\ddots&\dots&\ddots&\vdots\\
e^{-2j\pi (L-1)\frac{d}{\lambda}\sin(\theta_1)}&\dots&e^{-2j\pi (L-1)\frac{d}{\lambda}\sin(\theta_i)}&\dots&e^{-2j\pi (L-1)\frac{d}{\lambda}\sin(\theta_M)}
\end{bmatrix}
\end{gather*}
Пусть $h=-2j\pi\frac{d}{\lambda}$,
\begin{gather}
\frac{\partial A(\theta)}{\partial \theta_i} = \begin{bmatrix}
0&\dots&0&\dots&0\\
0&\dots&h\cos(\theta_i)e^{h \sin(\theta_i)}&\dots&0\\
\dots&\ddots&\dots&\ddots&\vdots\\
0&\dots&h(L-1)\cos(\theta_i)e^{h(L-1)\sin(\theta_i)}&\dots&0
\end{bmatrix}.
\end{gather}
Проблема заключается в том, что оптимизируемая функция $\tilde{Q}(\theta)$ -- вещественнозначная, $\theta \in \mathbf{R}^M$  а ее производная -- комплекснозначная. Будем находить экстремум по следующей схеме:
\begin{gather}
\hat{\theta}^{(\tau+1, k)}=\hat{\theta}^{(\tau+1, k-1)} - \eta Re(\grad \tilde{Q}(\theta^{(\tau)})),
\end{gather}
где $\hat{\theta}^{(\tau+1, k)}$ -- оценка $\theta^{\tau+1}$ на $k$-м шаге градиентного спуска, $\hat{\theta}^{(\tau+1, 0)} = \theta^{(\tau)}$.

\begin{center}
\fontsize{16}{20}\selectfont \color{red}{\textbf{Ранние неверные выкладки}}
\end{center}
\begin{gather*}
-(x_t-A S_t)^H \Gamma_n^{-1}(x_t-A S_t)-S_t^H\Gamma_s^{-1}S_t+x_t^H (A\Gamma_s  A^H + \Gamma_n)^{-1}x_t = -(S_t-\mu_{S_t|X_t})^H K_{S_t|X_t}^{-1}(S_t-\mu_{S_t|X_t})
\nonumber
\end{gather*}
\begin{gather*}
-x_t^H\Gamma_n^{-1}x_t + x_t^H\Gamma_n^{-1}A S_t + (A S_t)^H\Gamma_n^{-1}x_t - (A S_t)^H \Gamma_n^{-1} A S_t -S_t^H\Gamma_s^{-1}S_t \\
+x_t^H (A\Gamma_s  A^H + \Gamma_n)^{-1}x_t  =-S_t^H K_{S_t|X_t}^{-1}S_t + \mu_{S_t|X_t}^H  K_{S_t|X_t}^{-1}S_t   + S_t^H K_{S_t|X_t}^{-1}\mu_{S_t|X_t} - \mu_{S_t|X_t}^H K_{S_t|X_t}^{-1}\mu_{S_t|X_t}
\end{gather*}
\begin{equation*}
\left\{ \begin{aligned} 
 -x_t^H\Gamma_n^{-1}x_t + x_t^H (A\Gamma_s  A^H + \Gamma_n)^{-1}x_t = -\mu_{S_t|X_t}^H K_{S_t|X_t}^{-1}\mu_{S_t|X_t}\\
x_t^H\Gamma_n^{-1}A S_t = \mu_{S_t|X_t}^H  K_{S_t|X_t}^{-1}S_t \\
(A S_t)^H\Gamma_n^{-1}x_t = S_t^H K_{S_t|X_t}^{-1}\mu_{S_t|X_t} \\
-S_t^H\Gamma_s^{-1}S_t - (A S_t)^H \Gamma_n^{-1} A S_t = -S_t^H K_{S_t|X_t}^{-1}S_t
\end{aligned} \right.
\end{equation*}
\begin{equation*}
\left\{ \begin{aligned} 
 -x_t^H\Gamma_n^{-1}x_t + x_t^H (A\Gamma_s  A^H + \Gamma_n)^{-1}x_t = -\mu_{S_t|X_t}^H K_{S_t|X_t}^{-1}\mu_{S_t|X_t}\\
x_t^H\Gamma_n^{-1}A S_t = \mu_{S_t|X_t}^H  K_{S_t|X_t}^{-1}S_t \\
S_t^H A^H\Gamma_n^{-1}x_t = S_t^H K_{S_t|X_t}^{-1}\mu_{S_t|X_t} \\
-S_t^H\Gamma_s^{-1}S_t - S_t^H A^H \Gamma_n^{-1} A S_t = -S_t^H K_{S_t|X_t}^{-1}S_t
\end{aligned} \right.
\end{equation*}
\begin{equation*}
\left\{ \begin{aligned} 
 -x_t^H(\Gamma_n^{-1}- (A\Gamma_s  A^H+\Gamma_n)^{-1})x_t  = -\mu_{S_t|X_t}^H K_{S_t|X_t}^{-1}\mu_{S_t|X_t}\\
x_t^H\Gamma_n^{-1}A S_t = \mu_{S_t|X_t}^H  K_{S_t|X_t}^{-1}S_t \\
S_t^H A^H\Gamma_n^{-1}x_t = S_t^H K_{S_t|X_t}^{-1}\mu_{S_t|X_t} \\
-S_t^H(\Gamma_s^{-1}+A^H \Gamma_n^{-1} A)S_t = -S_t^H K_{S_t|X_t}^{-1}S_t
\end{aligned} \right.
\end{equation*}
Предполагая обратимость матрицы $\Gamma_s^{-1}+ A^H \Gamma_n^{-1} A$,
\begin{equation*}
-S_t^H(\Gamma_s^{-1}+A^H \Gamma_n^{-1} A)S_t = -S_t^H K_{S_t|X_t}^{-1}S_t  \Rightarrow \Gamma_s^{-1}+ A^H \Gamma_n^{-1} A= K_{S_t|X_t}^{-1} \Rightarrow K_{S_t|X_t} = (\Gamma_s^{-1}+ A^H \Gamma_n^{-1} A)^{-1}
\end{equation*}
\begin{gather*}
S_t^H A^H\Gamma_n^{-1}x_t = S_t^H K_{S_t|X_t}^{-1}\mu_{S_t|X_t}
\end{gather*}
Равенство выше должно выполняться для любых реализаций $S_t$, множитель $S_t^H$ является первым множителем в обоих произведениях, соответственно, равенство останется верным после удаления этих множителей.
\begin{gather*}
-A^H\Gamma_n^{-1}x_t = (\Gamma_s^{-1}+ A^H \Gamma_n^{-1} A)^{-1} \mu_{S_t|X_t} \Rightarrow \mu_{S_t|X_t} =- (\Gamma_s^{-1}+ A^H \Gamma_n^{-1} A) A^H\Gamma_n^{-1}x_t
\end{gather*}
\begin{equation}
\left\{ \begin{gathered} 
K_{S_t|X_t} = (\Gamma_s^{-1}+ A^H \Gamma_n^{-1} A)^{-1} \\
\mu_{S_t|X_t} =- (\Gamma_s^{-1}+ A^H \Gamma_n^{-1} A) A^H\Gamma_n^{-1}x_t
\end{gathered} \right.
\end{equation}
\end{document}
