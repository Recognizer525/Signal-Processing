\documentclass[11pt]{article}
\usepackage[english,russian]{babel}
\usepackage[utf8]{inputenc}
\usepackage[a4paper, left=2.5cm, right=1.5cm, top=2.5cm, bottom=2.5cm]{geometry}
\usepackage{animate} 
\usepackage{graphicx}
\usepackage{amsmath}
\usepackage{longtable}
\usepackage{amssymb}
\usepackage{physics}
\usepackage{tikz}
\usepackage{comment}
\usepackage{animate} 
\usepackage{graphicx}
\usepackage{amsmath}
\usepackage{longtable}
\usepackage{amssymb}
\usepackage{physics}
\usepackage{tikz}
\usepackage{comment}
\usepackage{colortbl}
\usepackage{xcolor}
\usepackage[normalem]{ulem}
\usepackage{float}
\usepackage{wrapfig}
\usepackage{cancel}
\usepackage{mathtools}
\usepackage[most]{tcolorbox}


\DeclarePairedDelimiter\ceil{\lceil}{\rceil}
 
\newcommand{\Expect}{\mathsf{M}}
\newcommand{\Var}{\mathsf{D}}
\newcommand{\Cov}{\mathsf{cov}}
\newcommand{\Norm}{\mathcal{N}}
\newcommand{\NormComplex}{\mathcal{CN}}
\newcommand{\Real}{\mathbb{R}}
\newcommand{\Int}{\mathbb{Z}}
\newcommand{\XSig}{\mathbf{x}}
\newcommand{\Ssig}{\mathbf{s}}
\newcommand{\Nsig}{\mathbf{n}}
\newcommand{\Rs}{\mathbf{R}_s}
\newcommand{\Rn}{\mathbf{R}_n}
\newcommand{\DK}{\mathbf{D}_{KL}}
\DeclarePairedDelimiterX{\infdivx}[2]{(}{)}{%
  #1\;\delimsize\|\;#2%
}
\DeclareMathOperator*{\argmax}{arg\,max}
\DeclareMathOperator*{\argmin}{arg\,min}
\newcommand{\infdiv}{D_{KL}\infdivx}
\newcommand\Fontvi{\fontsize{8.2}{7.2}\selectfont}
\newcommand\Fontvia{\fontsize{9}{8}\selectfont}
\newcommand\Fontvib{\fontsize{10.8}{9.6}\selectfont}
\newcommand\Fontvic{\fontsize{8.0}{7.0}\selectfont}
\newcommand{\myitem}{\item[\checkmark]}
%\newcommand{\myitem}{\item[\squares]}

\begin{document}
\begin{center}
\fontsize{20}{23}\selectfont \color{red}{\textbf{О добавлении неопределенности в случае комплексного нормального распределения}}
\end{center}
Комплексное нормальное распределение $CN(m_x,K_x)$  имеет следующую плотность:
\begin{gather}
f(x) = \frac{1}{|\pi K_x|} \cdot \exp\{-(x^H-m_x^H)K_x^{-1}(x-m_x)\},
\end{gather}
где $m_x$ имеет размерность $n\times 1$, $K_x$ имеет размерность $n\times n$.
Вынесем константу из под определителя, учитывая, что каждый столбец матрицы умножается на нее.
$$
f(x) = \frac{1}{\pi^n |K_x|} \cdot \exp\{-(x^H-m_x^H)K_x^{-1}(x-m_x)\},
$$
Пусть вектор $x$ составлен из элементов вектора $x_1$ и элементов вектора $x_2$:
$$
x=\begin{bmatrix}
x_1\\
x_2
\end{bmatrix}.
$$
Тогда, вектор $x_2$ имеет распределение
\begin{gather}
f(x_2) = \frac{1}{\pi^{n_2} |K_{x_2}|} \cdot \exp\{-(x_2^H-m_{x_2}^H)K_{x_2}^{-1}(x_2-m_{x_2})\}.
\end{gather}
Плотность вектора $x$ можно рассматривать как совместную плотность $x_1$, $x_2$: $f(x)=f(x_1,x_2).$
Условная плотность $x_1$ при условии $x_2$ должна иметь вид:
\begin{gather}
f(x_1|x_2)=\frac{f(x_1,x_2)}{f(x_2)},
\end{gather}
преобразуя выражение, получаем следующее:
\begin{gather}
f(x_1|x_2)=\frac{\frac{1}{\pi^n |K_x|} \cdot \exp\{-(x^H-m_x^H)K_x^{-1}(x-m_x)\}}{\frac{1}{\pi^{n_2} |K_{x_2}|} \cdot \exp\{-(x_2^H-m_{x_2}^H)K_{x_2}^{-1}(x_2-m_{x_2})\}}.
\end{gather}
Ковариацию вектора $x$ и обратную к ней матрицу представим в виде блочных матриц:
$$
K_X=\begin{bmatrix}
K_{11}&K_{12}\\
K_{21}&K_{22}
\end{bmatrix}, 
K_X^{-1}=\begin{bmatrix}
K^{11}&K^{12}\\
K^{21}&K^{22}
\end{bmatrix}.
$$

$$
f(x_1|x_2)=\frac{1}{\pi^{n-n_2}}\cdot \frac{|K_{22}|}{|K|}\cdot\frac{\exp\{-(x^H-m_x^H)K_x^{-1}(x-m_x)\}}{\exp\{-(x_2^H-m_{x_2}^H)K_{22}^{-1}(x_2-m_{x_2})\}}.
$$
Преобразуем дальше:
$$
f(x_1|x_2)=\frac{1}{\pi^{n-n_2}}\cdot \frac{|K_{22}|}{|K|}\cdot \exp[-(x^H-m_x^H)K_x^{-1}(x-m_x)+(x_2^H-m_{x_2}^H)K_{22}^{-1}(x_2-m_{x_2})]
$$
Выделим блоки $K_x^{-1}$:

\begin{equation*}
\begin{gathered}
f(x_1|x_2)=\frac{1}{\pi^{n-n_2}}\cdot \frac{|K_{22}|}{|K|}\cdot \exp \left[
-\left(\begin{bmatrix}
x_1^H\\
x_2^H
\end{bmatrix}-
\begin{bmatrix}
m_{x_1}^H\\
m_{x_2}^H
\end{bmatrix}\right)
\begin{bmatrix}
K^{11}&K^{12}\\
K^{21}&K^{22}
\end{bmatrix}
\left(\begin{bmatrix}
x_1\\
x_2
\end{bmatrix}-
\begin{bmatrix}
m_{x_1}\\
m_{x_2}
\end{bmatrix}\right) \right. \\
\left.+(x_2^H-m_{x_2}^H)K_{22}^{-1}(x_2-m_{x_2})\right]
\end{gathered}
\end{equation*}

\begin{gather*}
f(x_1|x_2)=\frac{1}{\pi^{n-n_2}}\cdot \frac{|K_{22}|}{|K|}\cdot \exp \left[-
\begin{bmatrix}
x_1^H-m_{x_1}^H\\
x_2^H-m_{x_2}^H
\end{bmatrix}
\begin{bmatrix}
K^{11}&K^{12}\\
K^{21}&K^{22}
\end{bmatrix}
\begin{bmatrix}
x_1-m_{x_1}\\
x_2-m_{x_2}
\end{bmatrix}
\right. \\
\left.
+(x_2^H-m_{x_2}^H)K_{22}^{-1}(x_2-m_{x_2})\right]
\end{gather*}



\begin{equation*}
\begin{gathered}
f(x_1|x_2)=\frac{1}{\pi^{n-n_2}}\cdot \frac{|K_{22}|}{|K|}\cdot \exp \left[-\left((x_1^H-m_{x_1}^H)K^{11}(x_1-m_{x_1})+(x_2^H-m_{x_2}^H)K^{21}(x_1-m_{x_1})+
\right. \right.
\\
\left.+(x_1^H-m_{x_1}^H)K^{12}(x_2-m_{x_2})+(x_2^H-m_{x_2}^H)K^{22}(x_2-m_{x_2})\right)+
\\
\left.
+(x_2^H-m_{x_2}^H)K_{22}^{-1}(x_2-m_{x_2})\right]
\end{gathered}
\end{equation*}
Теперь нужно определить, как вычислить элементы матрицы $K^{-1}$:
\begin{align}
\begin{bmatrix}
A& B\\
C& D
\end{bmatrix}^{-1}
=
\begin{bmatrix}
(A-BD^{-1}C)^{-1} & -(A-BD^{-1}C)^{-1}BD^{-1}\\
-D^{-1}C(A-BD^{-1}C)^{-1} & D^{-1}+D^{-1}C(A-BD^{-1}C)^{-1}BD^{-1}
\end{bmatrix},
\nonumber
\end{align}
таким образом:
\begin{align}
\begin{bmatrix}
K_{11} & K_{12}\\
K_{21} & K_{22}
\end{bmatrix}^{-1}
=
\begin{bmatrix}
(K_{11}-K_{12}K_{22}^{-1}K_{21})^{-1} & -(K_{11}-K_{12}K_{22}^{-1}K_{21})^{-1}K_{12}K_{22}^{-1}\\
-K_{22}^{-1}K_{21}(K_{11}-K_{12}K_{22}^{-1}K_{21})^{-1} & K_{22}^{-1}+K_{22}^{-1}K_{21}(K_{11}-K_{12}K_{22}^{-1}K_{21})^{-1}K_{12}K_{22}^{-1}
\end{bmatrix}
\nonumber.
\end{align}

\begin{equation*}
\begin{gathered}
f(x_1|x_2)=\frac{1}{\pi^{n-n_2}}\cdot \frac{|K_{22}|}{|K|}\cdot \exp \left[-\left((x_1^H-m_{x_1}^H)(K_{11}-K_{12}K_{22}^{-1}K_{21})^{-1}(x_1-m_{x_1})+ \right.\right. 
\\
+(x_2^H-m_{x_2}^H)(-K_{22}^{-1}K_{21}(K_{11}-K_{12}K_{22}^{-1}K_{21})^{-1})(x_1-m_{x_1})+
\\
+(x_1^H-m_{x_1}^H)(-(K_{11}-K_{12}K_{22}^{-1}K_{21})^{-1}K_{12}K_{22}^{-1})(x_2-m_{x_2})+
\\
\left.+(x_2^H-m_{x_2}^H)(K_{22}^{-1}+K_{22}^{-1}K_{21}(K_{11}-K_{12}K_{22}^{-1}K_{21})^{-1}K_{12}K_{22}^{-1})(x_2-m_{x_2})\right)+
\\
\left.+(x_2^H-m_{x_2}^H)K_{22}^{-1}(x_2-m_{x_2})\right]
\end{gathered}
\end{equation*}

\begin{equation*}
\begin{gathered}
f(x_1|x_2)=\frac{1}{\pi^{n-n_2}}\cdot \frac{|K_{22}|}{|K|}\cdot \exp \left[-\left((x_1^H-m_{x_1}^H)(K_{11}-K_{12}K_{22}^{-1}K_{21})^{-1}(x_1-m_{x_1})+ \right.\right.
\\
+(x_2^H-m_{x_2}^H)(-K_{22}^{-1}K_{21}(K_{11}-K_{12}K_{22}^{-1}K_{21})^{-1})(x_1-m_{x_1})+
\\
+(x_1^H-m_{x_1}^H)(-(K_{11}-K_{12}K_{22}^{-1}K_{21})^{-1}K_{12}K_{22}^{-1})(x_2-m_{x_2})+
\\
\left.\left.+(x_2^H-m_{x_2}^H)(K_{22}^{-1}K_{21}(K_{11}-K_{12}K_{22}^{-1}K_{21})^{-1}K_{12}K_{22}^{-1})(x_2-m_{x_2})\right)\right]
\end{gathered}
\end{equation*}

Определитель блочной матрицы вычисляется следующим образом:
\begin{gather}
\begin{vmatrix}
A & B \\
C & D
\end{vmatrix}
=
|D|\cdot |A-BD^{-1}C|.
\end{gather}
Соответственно, для ковариационной матрицы будет выполняться следующее:
\begin{align}
\begin{vmatrix}
K_{11} & K_{12} \\
K_{21} & K_{22}
\end{vmatrix}
=
|K_{22}|\cdot |K_{11}-K_{12}K_{22}^{-1}K_{21}|.
\nonumber
\end{align}



Используем этот факт, чтобы упростить соотношение определителей в условной плотности, заодно учтем, что $-K_{22}^{-1}K_{21}=(-K_{12}K_{22}^{-1})^H$:
\begin{equation*}
\begin{gathered}
f(x_1|x_2)=\frac{1}{\pi^{n-n_2}}\cdot \frac{1}{ |K_{11}-K_{12}K_{22}^{-1}K_{21}|}\cdot \exp \left[-\left((x_1^H-m_{x_1}^H)(K_{11}-K_{12}K_{22}^{-1}K_{21})^{-1}(x_1-m_{x_1})+ \right.\right.
\\
+(x_2^H-m_{x_2}^H)(-K_{12}K_{22}^{-1})^H(K_{11}-K_{12}K_{22}^{-1}K_{21})^{-1}(x_1-m_{x_1})+
\\
+(x_1^H-m_{x_1}^H)(K_{11}-K_{12}K_{22}^{-1}K_{21})^{-1}(-K_{12}K_{22}^{-1})(x_2-m_{x_2})+
\\
\left.\left.+(x_2^H-m_{x_2}^H)(-K_{12}K_{22}^{-1})^H(K_{11}-K_{12}K_{22}^{-1}K_{21})^{-1}(-K_{12}K_{22}^{-1})(x_2-m_{x_2})\right)\right]
\end{gathered}
\end{equation*}

Немного перегруппируем составляющие показателя степени:
\begin{equation}
\begin{gathered}
f(x_1|x_2)=\frac{1}{\pi^{n-n_2}}\cdot \frac{1}{ |K_{11}-K_{12}K_{22}^{-1}K_{21}|}\cdot \exp \left[-\left(((x_1^H-m_{x_1}^H)+(x_2^H-m_{x_2}^H) \right.\right. \\
\cdot(-K_{12}K_{22}^{-1})^H)(K_{11}-K_{12}K_{22}^{-1}K_{21})^{-1}(x_1-m_{x_1})+ \\
\left.\left.+(x_1^H-m_{x_1}^H+(x_2^H-m_{x_2}^H)(-K_{12}K_{22}^{-1})^H)(K_{11}-K_{12}K_{22}^{-1}K_{21})^{-1}(-K_{12}K_{22}^{-1})(x_2-m_{x_2})\right)\right]
\end{gathered}
\end{equation}
Еще раз перегруппируем составляющие показателя степени:
\begin{equation}
\begin{gathered}
f(x_1|x_2)=\frac{1}{\pi^{n-n_2}}\cdot \frac{1}{ |K_{11}-K_{12}K_{22}^{-1}K_{21}|}\cdot \exp \left[-\left(((x_1^H-m_{x_1}^H)+(x_2^H-m_{x_2}^H)\cdot(-K_{12}K_{22}^{-1})^H) \right.\right. \\
\left.\left. \cdot(K_{11}-K_{12}K_{22}^{-1}K_{21})^{-1}((x_1-m_{x_1})+(-K_{12}K_{22}^{-1})(x_2-m_{x_2}))\right)\right]
\end{gathered}
\end{equation}
Тогда,
\begin{equation}
\left\{ \begin{gathered} 
K_{x_1|x_2} = K_{11}-K_{12}K_{22}^{-1}K_{21} \\
m_{x_1|x_2} = m_{x_1} + K_{12}K_{22}^{-1}\cdot(x_2-m_{x_2}),
\end{gathered} \right.
\end{equation}
\end{document}