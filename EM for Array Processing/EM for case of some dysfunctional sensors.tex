\documentclass[11pt]{article}
\usepackage[english,russian]{babel}
\usepackage[utf8]{inputenc}
\usepackage[a4paper, left=2.5cm, right=1.5cm, top=2.5cm, bottom=2.5cm]{geometry}
\usepackage{animate} 
\usepackage{graphicx}
\usepackage{amsmath}
\usepackage{longtable}
\usepackage{amssymb}
\usepackage{physics}
\usepackage{tikz}
\usepackage{comment}
\usepackage{animate} 
\usepackage{graphicx}
\usepackage{amsmath}
\usepackage{longtable}
\usepackage{amssymb}
\usepackage{physics}
\usepackage{tikz}
\usepackage{comment}
\usepackage{colortbl}
\usepackage{xcolor}
\usepackage[normalem]{ulem}
\usepackage{float}
\usepackage{wrapfig}
\usepackage{cancel}
\usepackage{mathtools}
\usepackage[most]{tcolorbox}
\usepackage[mathscr]{euscript}

\DeclarePairedDelimiter\ceil{\lceil}{\rceil}
 
\newcommand{\Expect}{\mathsf{M}}
\newcommand{\Var}{\mathsf{D}}
\newcommand{\Cov}{\mathsf{cov}}
\newcommand{\Norm}{\mathcal{N}}
\newcommand{\NormComplex}{\mathcal{CN}}
\newcommand{\Real}{\mathbb{R}}
\newcommand{\Int}{\mathbb{Z}}
\newcommand{\XSig}{\mathbf{x}}
\newcommand{\Ssig}{\mathbf{s}}
\newcommand{\Nsig}{\mathbf{n}}
\newcommand{\Rs}{\mathbf{R}_s}
\newcommand{\Rn}{\mathbf{R}_n}
\newcommand{\DK}{\mathbf{D}_{KL}}
\DeclarePairedDelimiterX{\infdivx}[2]{(}{)}{%
  #1\;\delimsize\|\;#2%
}
\DeclareMathOperator*{\argmax}{arg\,max}
\DeclareMathOperator*{\argmin}{arg\,min}
\newcommand{\infdiv}{D_{KL}\infdivx}
\newcommand\Fontvi{\fontsize{8.2}{7.2}\selectfont}
\newcommand\Fontvia{\fontsize{9}{8}\selectfont}
\newcommand\Fontvib{\fontsize{10.8}{9.6}\selectfont}
\newcommand\Fontvic{\fontsize{8.0}{7.0}\selectfont}
\newcommand{\myitem}{\item[\checkmark]}
%\newcommand{\myitem}{\item[\squares]}

\begin{document}
\begin{center}
\fontsize{20}{23}\selectfont \color{red}{\textbf{ЕМ-алгоритм для оценки направления прибытия сигнала}}
\end{center}
Введем некоторые условные обозначения:
\begin{itemize}
\item
$\theta$ -- вектор направлений прибытия сигнала (DOA);
\item
$\tau$ -- итерация ЕМ-алгоритма, начальная оценка параметров $\theta$;
\item
$t$ -- момент времени (а заодно и номер кадра (snapshot));
\item
$L$ -- число датчиков;
\item
$M$ -- число источников (источники разделяют общую длину центральной волны $\chi$);
\item
$G$ -- число независимых кадров/снимков (snapshot), сделанных в разные моменты времени;
\item
$S$ -- набор сигналов (случайная величина), испускаемых источниками в моменты времени $t=\overline{1,G}$, $S_t$ соответствует сигналу в момент времени $t$;
\item
$s$ -- набор сигналов (реализация), испускаемых источниками в моменты времени $t=\overline{1,G}$, $s_t$ соответствует сигналу в момент времени $t$;
\item
$N$ -- набор шумов (случайная величина), связанных с датчиками в моменты времени $t=\overline{1,G}$, $N_t$ соответствует шуму в момент времени $t$;
\item
$n$ -- набор шумов (реализация), связанных с датчиками в моменты времени $t=\overline{1,G}$, $n_t$ соответствует шуму в момент времени $t$;
\item
$X$ -- набор сигналов (случайная величина), полученных датчиками в моменты времени $t=\overline{1,G}$, $X_t$ соответствует сигналу в момент времени $t$;
\item
$x$ -- набор сигналов (реализация), полученных датчиками в моменты времени $t=\overline{1,G}$, $x_t$ соответствует сигналу в момент времени $t$;
\item
$X_o$ -- наблюдаемая часть (случайная величина) $X$; 
\item
$x_o$ -- наблюдаемая часть (реализация) $X$;
\item
$X_m$ -- ненаблюдаемая часть (случайная величина) $X$;
\item
$x_m$ -- ненаблюдаемая часть (реализация) $X$;
\item
$Z$ -- латентные переменные (случайная величина) ($S, X_m$);
\item
$z$ -- латентные переменные (реализация) ($s, x_m$);
\item
$\psi$ -- параметры комплексного нормального распределения $X_m$;
\item
$\Omega$ -- $(\psi, \theta)$;
\item
$O_{D_1 \times D_2}$ -- нулевая матрица размера $D_1 \times D_2$;
\item
Итоговый сигнал, получаемый массивом датчиков:
\begin{equation}
\begin{gathered}
X_t=A(\theta)S_t+N_t,
\end{gathered}
\end{equation}
где $S_t \sim CN(0,\Gamma_s),t=\overline{1,G}$, $N_t \sim CN(0,\Gamma_n), t=\overline{1,G}$, $S_t$ имеет размер $M \times 1$,  $N_t$ имеет размер $L \times 1$, $\theta=[\theta_1,...,\theta_M]$ -- вектор направлений прибытия сигнала, $A(\theta)$ (далее - $A$) представляет собой матрицу управляющих векторов размера $L \times M$, $\Gamma_s$ и $\Gamma_n$ предполагаются диагольными.
\begin{gather}
A(\theta) = \begin{bmatrix}
1&1&\dots&1\\
e^{-2j\pi \frac{d}{\lambda}sin(\theta_1)}& e^{-2j\pi \frac{d}{\lambda}sin(\theta_2)}&\dots&e^{-2j\pi \frac{d}{\lambda}sin(\theta_M)}\\
\dots&\dots&\ddots&\vdots\\
e^{-2j\pi (L-1) \frac{d}{\lambda}sin(\theta_1)}& e^{-2j\pi (L-1) \frac{d}{\lambda}sin(\theta_2)}&\dots&e^{-2j\pi (L-1) \frac{d}{\lambda}sin(\theta_M)}\\
\end{bmatrix}.
\nonumber
\end{gather}
\end{itemize}
Воспользуемся ЕМ-алгоритмом для того, чтобы определить значения параметров $\theta$, значения сигналов $S_t, t=\overline{1,G}$ рассматриваются как латентные переменные. 
Пусть $X$, $S$ и $N$ набор итоговых сигналов полученных $L$ датчиками за моменты времени $t=\overline{1,G}$ и набор выпущенных $M$ источниками сигналов и набор шумов за моменты времени $t=\overline{1,G}$, соответственно. $X$, $S$ и $N$ представляют из себя матрицы размеров $G \times L$, $G \times M$ и $G \times L$ соответственно.
\begin{center}
\fontsize{16}{20}\selectfont \color{red}{\textbf{Е-шаг}}
\end{center}
Требуется найти апостериорное распределение $P(Z|X=x,\theta)$, воспользуемся формулой Байеса:
\begin{gather}
P(Z|X=x,\Omega) = \frac{P(X, S|\Omega)}{P(X_m, S|\Omega)} = \frac{P(X_o, X_m, S|\Omega)}{P(X_m, S|\Omega)} = \frac{P(X_o, X_m, S|\Omega)}{P(X_m|S=s, \Omega)P(S|\Omega)} 
\end{gather}
\begin{gather}
P(S|\Omega) = \prod_{t=1}^G \frac{1}{\pi^M |\Gamma_s|}e^{-S_t^H\Gamma_s^{-1}S_t},
\end{gather}
\begin{gather*}
X_t = AS_t + N_t \\
X_t \sim CN(0, A\Gamma_s A^H + \Gamma_n)
\end{gather*}
\begin{gather}
P(X|\theta) = \prod_{t=1}^G \frac{1}{\pi^L |A\Gamma_s  A^H + \Gamma_n|}e^{-X_t^H (A\Gamma_s A^H + \Gamma_n)^{-1}X_t},
\end{gather}
Теперь следует определиться с тем, каким будет условное распределение $P(X|S=s, \theta)$
\begin{gather*}
X_t|S_t=s_t, \theta \, \sim CN(A s_t, \Gamma_n)
\end{gather*}
\begin{gather}
P(X|S=s,\theta) = \prod_{t=1}^G \frac{1}{\pi^L |\Gamma_n|}e^{-(X_t-A s_t)^H \Gamma_n^{-1}(X_t-A s_t)},
\end{gather}
\end{document}
